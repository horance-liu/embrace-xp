\begin{savequote}[45mm]
\ascii{Any fool can write code that a computer can understand. Good programmers write code that humans can understand.}
\qauthor{\ascii{- Martin Flower}}
\end{savequote}

\chapter{异常处理} 
\label{ch:except-handle}

\begin{content}

\end{content}

\section{断言失败}

\begin{content}

至此,框架不能处理任何异常逻辑。当用户调用\ascii{ASSERT\_EQ}失败时,抛出\ascii{AssertionError}异常,框架捕获该异常,标记该测试用例为\ascii{Failure};如果抛出其他异常,并被\ascii{mars}框架捕获,则标记该测试用例为\ascii{Error}。显式区分这两种异常类型,使得用户可以审查自己的测试用例,并修复失败的用例。

\subsection{测试用例}

当用例的断言失败,打印诸如如下详细信息,提示用户修复失败的用例。

\begin{leftbar}
 \begin{python}[caption={测试失败}]
/home/horance/code/cpp/mars/test/mars/core/TestCaseSpec.cc:33
expected value == 2, but got 0
 \end{python}
\end{leftbar}

接下来,构建失败的用例,在\ascii{runTest}时抛出异常\ascii{AssertionError},模拟断言失败的场景。

\begin{leftbar}
 \begin{c++}[caption={\ttfamily{test/mars/TestCaseSpec.cc}}]
#include <gtest/gtest.h>
#include "mars/core/TestCase.h"
#include "mars/except/AssertionError.h"

namespace {
  struct TestCaseSpec : testing::Test {
  protected:
    void run(::Test& test) {
      test.run(result);
    }

  protected:
    TestResult result;
  };

  struct AssertionFailedTest : TestCase {
    void runTest() override {
      throw AssertionError("test.cpp:57", "expected value == 2, but got 3");
    }
  };
}

TEST_F(TestCaseSpec, throw_assertion_error_on_run_test) {
  AssertionFailedTest test;
  run(test);

  ASSERT_EQ(1, result.failCount());
}
 \end{c++}
\end{leftbar}

\subsection{异常类}

为了通过编译,引入\ascii{AssertionError}的异常类。它携带了断言失败处的文件名与行号\ascii{(src)},及其用例断言失败的具体原因\ascii{(msg)}。

\begin{leftbar}
 \begin{c++}[caption={\ttfamily{include/mars/except/AssertionError.h}}]
#include <string>
#include <exception>

struct AssertionError : std::exception {
  AssertionError(const std::string& src, const std::string& msg);
  ~AssertionError() noexcept = default;

  const char* what() const noexcept;

private:
  std::string msg;
};
 \end{c++}
\end{leftbar}

\ascii{AssertionError}实现如下。

\begin{leftbar}
 \begin{c++}[caption={\ttfamily{src/mars/except/AssertionError.cc}}]
#include "mars/except/AssertionError.h"

AssertionError::AssertionError(const std::string& src,
  const std::string& msg) : msg(src + "\n" + msg) {
}

const char* AssertionError::what() const noexcept {
  return msg.c_str();
}
 \end{c++}
\end{leftbar}

\subsection{失败计数器}

\ascii{TestResult}新增名为\ascii{numOfFails}的计数器。

\begin{leftbar}
 \begin{c++}[caption={\ttfamily{include/mars/core/TestResult.h}}]
struct TestResult {
  TestResult();

  int failCount() const;
  void onFail();

private:
  int numOfFails;
};
 \end{c++}
\end{leftbar}

在实现文件中,实现失败计数器的初始化,累加,与查询逻辑。

\begin{leftbar}
 \begin{c++}[caption={\ttfamily{src/mars/core/TestResult.cc}}]
#include "mars/core/TestResult.h"

TestResult::TestResult() : numOfFails(0) {
}

int TestResult::failCount() const {
  return numOfFails;
}

void TestResult::onFail() {
  numOfFails++;
}
 \end{c++}
\end{leftbar}

\subsection{捕获异常}

当执行\ascii{TestCase::run}时,捕获异常并通过调用\ascii{TestResult::fail},累加计数器\ascii{numOfFails}。

\begin{leftbar}
 \begin{c++}[caption={\ttfamily{src/mars/core/TestCase.cc}}]
#include "mars/core/TestCase.h"
#include "mars/core/TestResult.h"
#include "mars/except/AssertionError.h"

void TestCase::run(TestResult& result) {
  setUp();

  try {
    runTest();
  } catch (const AssertionError&) {
    result.onFail();
  }

  tearDown();
}

// ...
 \end{c++}
\end{leftbar}

至此,测试通过。

\section{前置失败}

\ascii{TestCase}执行时,还未执行至\ascii{runTest}时,执行\ascii{setUp}可能就提前失败了。

\subsection{测试用例}

构建失败的用例,模拟前置失败的场景。

\begin{leftbar}
 \begin{c++}[caption={\ttfamily{test/mars/TestCaseSpec.cc}}]
namespace {
  struct AssertionFailedOnSetUpTest : TestCase {
    bool wasRun = false;

  private:
    void setUp() override {
      throw AssertionError("test.cpp:57", "expected value == 2, but got 3");
    }

    void runTest() override {
      wasRun = true;
    }
  };
}

TEST_F(TestCaseSpec, throw_assertion_error_on_setup) {
  AssertionFailedOnSetUpTest test;
  run(test);

  ASSERT_EQ(1, result.failCount());
  ASSERT_FALSE(test.wasRun);
}
 \end{c++}
\end{leftbar}

\subsection{捕获异常}

为了通过测试,当执行\ascii{TestCase::setUp}时,捕获异常并通知\ascii{TestResult},然后立即终止用例的执行;毕竟环境没准备好,强制执行用例毫无意义。

\begin{leftbar}
 \begin{c++}[caption={\ttfamily{src/mars/core/TestCase.cc}}]
#include "mars/core/TestCase.h"
#include "mars/core/TestResult.h"
#include "mars/except/AssertionError.h"

void TestCase::run(TestResult& result) {
  bool succ = false;
  try {
    setUp();
    succ = true;
  } catch (const AssertionError&) {
    result.onFail();
  }

  if (succ) {
    try {
      runTest();
    } catch (const AssertionError&) {
      result.onFail();
    }
  }

  tearDown();
}
 \end{c++}
\end{leftbar}

至此,测试通过。

\section{后置失败}

\ascii{TestCase::runBare}执行时,执行完\ascii{setUp, runTest}成功之后,接着执行\ascii{tearDown}时失败。

\subsection{测试用例}

构建失败的用例,模拟后置失败的场景。

\begin{leftbar}
 \begin{c++}[caption={\ttfamily{test/mars/TestCaseSpec.cc}}]
namespace {
  struct AssertionFailedOnTearDownTest : TestCase {
    void tearDown() override {
      throw AssertionError("test.cpp:57", "expected value == 2, but got 3");
    }
  };
}

TEST_F(TestCaseSpec, throw_assertion_error_on_tear_down) {
  AssertionFailedOnTearDownTest test;
  run(test);

  ASSERT_EQ(1, result.failCount());
}
 \end{c++}
\end{leftbar}

\subsection{捕获异常}

当执行\ascii{TestCase::tearDown}时,捕获异常并通知\ascii{TestResult}。但是,无论\ascii{setUp, runTest}成败与否,\ascii{tearDown}都要强制执行,保证资源的安全释放。

\begin{leftbar}
 \begin{c++}[caption={\ttfamily{src/mars/core/TestCase.cc}}]
#include "mars/core/TestCase.h"
#include "mars/core/TestResult.h"
#include "mars/except/AssertionError.h"

void TestCase::run(TestResult& result) {
  bool succ = false;
  try {
    setUp();
    succ = true;
  } catch (const AssertionError&) {
    result.onFail();
  }

  if (succ) {
    try {
      runTest();
    } catch (const AssertionError&) {
      result.onFail();
    }
  }

  try {
    tearDown();
  } catch (const AssertionError&) {
    result.onFail();
  }
}
 \end{c++}
\end{leftbar}

至此,测试通过。

\section{错误:抛出std::exception}

当抛出\ascii{AssertionError},称之为\ascii{Failure};当抛出其他异常,称之为\ascii{Error}。

\subsection{测试用例}

构建失败的用例,模拟其他异常抛出的场景。

\begin{leftbar}
 \begin{c++}[caption={\ttfamily{test/mars/TestCaseSpec.cc}}]
namespace {
  struct StdExceptionTest : TestCase {
    void runTest() override {
      throw std::exception();
    }
  };
}

TEST_F(TestCaseSpec, throw_std_exception_on_run_test) {
  StdExceptionTest test;
  run(test);

  ASSERT_EQ(0, result.failCount());
  ASSERT_EQ(1, result.errorCount());
}
 \end{c++}
\end{leftbar}

\subsection{错误计数器}

增加查询函数\ascii{TestResult::errorCount}查询函数,错误监听接口\ascii{TestResult::onError},及其相应的错误计数器\ascii{numOfErrors}。

\begin{leftbar}
 \begin{c++}[caption={\ttfamily{include/mars/core/TestResult.h}}]
struct TestResult {
  TestResult();

  int failCount() const;
  int errorCount() const;

  void onFail();
  void onError();

private:
  int numOfFails;
  int numOfErrors;
};
 \end{c++}
\end{leftbar}

在实现文件中,依葫芦画瓢,实现\ascii{TestResult::errorCount, TestResult::onError}。

\begin{leftbar}
 \begin{c++}[caption={\ttfamily{src/mars/core/TestResult.cc}}]
#include "mars/core/TestResult.h"

TestResult::TestResult()
  : numOfFails(0), numOfErrors(0) {
}

int TestResult::failCount() const {
  return numOfFails;
}

int TestResult::errorCount() const {
  return numOfErrors;
}

void TestResult::onFail() {
  numOfFails++;
}

void TestResult::onError() {
  numOfErrors++;
}
 \end{c++}
\end{leftbar}

\subsection{捕获异常}

当执行\ascii{TestCase::runTest}时,捕获\ascii{std::exception}异常,并通知\ascii{TestResult::error}累加计数器。

\begin{leftbar}
 \begin{c++}[caption={\ttfamily{src/mars/core/TestCase.cc}}]
#include "mars/core/TestCase.h"
#include "mars/core/TestResult.h"
#include "mars/except/AssertionError.h"

void TestCase::run(TestResult& result) {
  bool succ = false;
  try {
    setUp();
    succ = true;
  } catch (const AssertionError&) {
    result.onFail();
  }

  if (succ) {
    try {
      runTest();
    } catch (const AssertionError&) {
      result.onFail();
    } catch (const std::exception&) {
      result.onError();
    }
  }

  try {
    tearDown();
  } catch (const AssertionError&) {
    result.onFail();
  }
}
 \end{c++}
\end{leftbar}

\section{未知错误}

当抛出其他未知的异常,测试框架需要捕获,并标记为\ascii{Error},提示用户被测系统存在严重的漏洞。

\subsection{测试用例}

\begin{leftbar}
 \begin{c++}[caption={\ttfamily{test/mars/TestCaseSpec.cc}}]
namespace {
  struct UnknownException {};

  struct UnknownExceptionTest : TestCase {
    void runTest() override {
      throw UnknownException();
    }
  };
}

TEST_F(TestCaseSpec, throw_unknown_exception_on_run_test) {
  UnknownExceptionTest test;
  run(test);

  ASSERT_EQ(0, result.failCount());
  ASSERT_EQ(1, result.errorCount());
}
 \end{c++}
\end{leftbar}

\subsection{捕获异常}

当执行\ascii{TestCase::runTest}时,捕获未知的异常,并通知\ascii{TestResult::error}累加计数器。

\begin{leftbar}
 \begin{c++}[caption={\ttfamily{src/mars/core/TestCase.cc}}]
#include "mars/core/TestCase.h"
#include "mars/core/TestResult.h"
#include "mars/except/AssertionError.h"

void TestCase::run(TestResult& result) {
  bool succ = false;
  try {
    setUp();
    succ = true;
  } catch (const AssertionError&) {
    result.onFail();
  }

  if (succ) {
    try {
      runTest();
    } catch (const AssertionError&) {
      result.onFail();
    } catch (const std::exception&) {
      result.onError();
    } catch (...) {
      result.onError();
    }    
  }

  try {
    tearDown();
  } catch (const AssertionError&) {
    result.onFail();
  }
}
 \end{c++}
\end{leftbar}

测试通过。

\section{完备的异常处理}

依次构建测试用例,小步快走,驱动出完备的异常处理逻辑。

\begin{leftbar}
 \begin{c++}[caption={\ttfamily{src/mars/core/TestCase.cc}}]
#include "mars/core/TestCase.h"
#include "mars/core/TestResult.h"
#include "mars/except/AssertionError.h"

void TestCase::run(TestResult& result) {
  bool succ = false;
  try {
    setUp();
    succ = true;
  } catch (const AssertionError&) {
    result.onFail();
  } catch (const std::exception&) {
    result.onError();
  } catch (...) {
    result.onError();
  }

  if (succ) {
    try {
      runTest();
    } catch (const AssertionError&) {
      result.onFail();
    } catch (const std::exception&) {
      result.onError();
    } catch (...) {
      result.onError();
    }
  }

  try {
    tearDown();
  } catch (const AssertionError&) {
    result.onFail();
  } catch (const std::exception&) {
    result.onError();
  } catch (...) {
    result.onError();
  }
}

// ...
 \end{c++}
\end{leftbar}

\ascii{TestCase::run}实现不仅存在重复逻辑,而且异常臃肿,异常丑陋。此时不重构,更待何时。

\subsection{消除重复}

观察\ascii{TestCase::setUp, TestCase::runTest, TestCase::tearDown}的共性,它们拥有相同的函数签名,可以使用指向成员函数的指针挖掘它们的共性。

\begin{leftbar}
 \begin{c++}
using Method = void(TestCase::*)();
 \end{c++}
\end{leftbar}

提取公共函数\ascii{TestCase::protect},它返回\ascii{bool}类型。首先,在头文件中声明私有函数\ascii{TestCase::protect}。

\begin{leftbar}
 \begin{c++}[caption={\ttfamily{include/mars/core/TestCase.h}}]
#include "mars/core/Test.h"

struct TestCase : Test {
  using Test::Test;

private:
  int countTestCases() const override;
  void run(TestResult&) override;

private:
  virtual void setUp() {}
  virtual void runTest() {}
  virtual void tearDown() {}

private:
  using Method = void(TestCase::*)();
  bool protect(TestResult& result, Method method);
};
 \end{c++}
\end{leftbar}

在实现文件中,实现异常处理逻辑的代码复用。

\begin{leftbar}
 \begin{c++}[caption={\ttfamily{src/mars/core/TestCase.cc}}]
#include "mars/core/TestCase.h"
#include "mars/core/TestResult.h"
#include "mars/except/AssertionError.h"

bool TestCase::protect(TestResult& result, Method method) {
  bool succ = false;
  try {
    (this->*method)();
    succ = true;
  } catch (const AssertionError&) {
    result.onFail();
  } catch (const std::exception&) {
    result.onError();
  } catch (...) {
    result.onError();
  }
  return succ;
}

void TestCase::run(TestResult& result) {
  if (protect(result, &TestCase::setUp)) {
    protect(result, &TestCase::runTest);
  }
  protect(result, &TestCase::tearDown);
}

int TestCase::countTestCases() const {
  return 1;
}
 \end{c++}
\end{leftbar}

\end{content}

