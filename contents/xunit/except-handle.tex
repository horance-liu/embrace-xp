\begin{savequote}[45mm]
\ascii{Any fool can write code that a computer can understand. Good programmers write code that humans can understand.}
\qauthor{\ascii{- Martin Flower}}
\end{savequote}

\chapter{异常处理} 
\label{ch:except-handle}

\begin{content}

本章重点讨论异常信息、异常检测、异常处理的实现机制。详细探讨增加异常机制后,保持设计与实现简洁性的相关技术。通过异常处理机制的开发,深入\ascii{TDD}、重构、软件设计的深水区,领悟软件之美。

\end{content}

\section{概念}

\begin{content}

\subsection{异常类型}

通过定制\ascii{AssertionError}异常,区分\emph{失败}与\emph{错误},利于用户判断测试执行失败的具体原因。如果是\emph{失败},则说明产品实现与测试用例预期不符。如果是\emph{错误},则说明产品实现存在严重缺陷,亟待解决。

\subsubsection{失败: Failure}

当调用诸如\ascii{ASSERT\_EQ, ASSERT\_TRUE}等函数判断测试结果是否与预期相符时,如果断言失败则抛出\ascii{AssertionError}异常。\ascii{xUnit Mars}框架捕获该异常,标记该测试用例为\emph{失败}\ascii{(Failure)}。

\subsubsection{错误: Error}

如果\ascii{xUnit Mars}框架捕获非\ascii{AssertionError}异常时,则标记该测试用例为\emph{错误}\ascii{(Error)}。\emph{错误}意味着产品实现存在严重缺陷,在发布生产环境前,必须拥有有效的保护机制。否则,将导致不可预期的行为发生。

\subsection{异常信息}

当某个测试用例执行失败,用户期望终端打印如下信息,方便用户快速修复失败的用例。

\begin{nodiff}{断言失败时,输出异常信息}
 \begin{c++}
assertion fail in the runTest
/home/horance/code/cpp/cut/test/hamcrest/core/ComparatorTest.cpp:12
Expected: a value eq <254: int>
     but: <255: int> eq <254: int> got false
 \end{c++}
\end{nodiff}

异常信息包括如下几个重要逻辑信息:

\begin{enum}
  \eitem{\ascii{Why}: 指明用例失败的原因}
    \begin{enum}
      \eitem{\ascii{assertion fail}: 断言失败}
      \eitem{\ascii{uncaught std::exception}: 产品实现抛出\ascii{std::exception}或子异常}
      \eitem{\ascii{uncaught unknown exception}: 产品实现抛出其他未知异常}
    \end{enum}
  \eitem{\ascii{Where}: 异常抛出的地点}
    \begin{enum}
      \eitem{\ascii{in the setUp}}
      \eitem{\ascii{in the runTest}}
      \eitem{\ascii{in the tearDown}}
    \end{enum}
  \eitem{\ascii{What}: 异常携带的详细信息}
    \begin{enum}
      \eitem{\ascii{源文件}}
      \eitem{\ascii{行号}}
      \eitem{\ascii{异常详细信息}: \ascii{what}成员函数返回值}
    \end{enum}
\end{enum}

\subsection{测试夹具}

本章后文所有测试用例,都基于相同的测试夹具。为了避免在每个测试用例中重复说明,此处给出该测试夹具的公共实现。后文不再重复显式给出\ascii{TestCaseSpec}的具体实现,也不会重复包含此处的三个头文件。

\begin{nodiff}{test/mars/core/TestCaseSpec.cc}
 \begin{c++}
#include <mars/core/TestCase.h>
#include <mars/core/TestResult.h>
#include <gtest/gtest.h>

namespace {
  struct TestCaseSpec : testing::Test {
  protected:
    void run(::Test& test) {
      test.run(result);
    }

  protected:
    TestResult result;
  };
}
 \end{c++}
\end{nodiff}

\end{content}

\section{失败}

\begin{content}

\subsection{断言失败}

构建一个失败的用例,在\ascii{runTest}时抛出异常\ascii{AssertionError},模拟断言失败的场景。

\begin{nodiff}{test/mars/core/TestCaseSpec.cc}
 \begin{c++}
#include <mars/except/AssertionError.h>

namespace {
  struct FailureOnRunningTest : TestCase {
    void runTest() override {
      throw AssertionError("product.cc:57", "expected value == 2, but got 3");
    }
  };
}

TEST_F(TestCaseSpec, throw_assertion_error_on_running_test) {
  FailureOnRunningTest test;
  run(test);

  ASSERT_EQ(1, result.failCount());
}
 \end{c++}
\end{nodiff}

\subsubsection{通过编译}

为了通过编译,引入异常类\ascii{AssertionError}。它的构造函数携带了断言失败处的文件名与行号\ascii{(src)},及其用例断言失败的具体原因\ascii{(msg)}。

\begin{nodiff}{include/mars/except/AssertionError.h}
 \begin{c++}
#include <string>
#include <exception>

struct AssertionError : std::exception {
  AssertionError(const std::string& src, const std::string& msg);
};
 \end{c++}
\end{nodiff}

在\ascii{TestResult}中公开查询接口\ascii{failCount}。

\begin{nodiff}{include/mars/core/TestResult.h}
 \begin{c++}
struct TestResult {
  int failCount() const;
};
 \end{c++}
\end{nodiff}

至此,编译通过。

\subsubsection{通过测试}

伪实现\ascii{TestResult::failCount},直接返回\ascii{1}。

\begin{nodiff}{include/mars/core/TestResult.cc}
 \begin{c++}
#include <mars/core/TestResult.h>

int TestResult::failCount() const {
  return 1;
}
 \end{c++}
\end{nodiff}

空实现\ascii{AssertionError}的构造函数。

\begin{nodiff}{include/mars/except/AssertionError.cc}
 \begin{c++}
#include <mars/except/AssertionError.h>

AssertionError::AssertionError
  ( const std::string& src
  , const std::string& msg) {}
 \end{c++}
\end{nodiff}

本期望顺利通过测试,但\ascii{Google Test}执行失败。

\begin{nodiff}{Google Test执行失败}
 \begin{c++}
[ RUN      ] TestCaseSpec.throw_assertion_error_on_run_test
unknown file: Failure
C++ exception with description "std::exception" thrown in the test body.
[  FAILED  ] TestCaseSpec.throw_assertion_error_on_run_test (2 ms)
 \end{c++}
\end{nodiff}

原因非常简单,\ascii{Google Test}并不认识\ascii{AssertionError}。此时需要\ascii{xUnit Mars}捕获此异常,并在其内部完成统计工作。

\subsubsection{捕获异常}

当执行\ascii{TestCase::run}时,捕获异常\ascii{AssertionError},并通过调用\ascii{TestResult::addFailure}接口,通知\ascii{TestResult}完成错误统计任务。

\begin{nodiff}{src/mars/core/TestCase.cc}
 \begin{c++}
#include <mars/core/TestCase.h>
#include <mars/core/TestResult.h>
#include <mars/except/AssertionError.h>

void TestCase::run(TestResult& result) {
  setUp();
  try {
    runTest();
  } catch (const AssertionError&) {
    result.addFailure();
  }
  tearDown();
}
 \end{c++}
\end{nodiff}

\subsubsection{失败计数器}

为了通过测试,在\ascii{TestResult}中增加默认构造函数、\emph{失败计数器}监听接口\ascii{addFailure}、成员变量\ascii{numOfFails}。

\begin{nodiff}{include/mars/core/TestResult.cc}
 \begin{c++}
struct TestResult {
  TestResult();

  void addFailure();
  int failCount() const;

private:
  int numOfFails;
};
 \end{c++}
\end{nodiff}

在实现文件中,实现\emph{失败计数器}的初始化、累加,及其查询逻辑。

\begin{nodiff}{src/mars/core/TestResult.cc}
 \begin{c++}
#include <mars/core/TestResult.h>

TestResult::TestResult() : numOfFails(0) {
}

void TestResult::addFailure() {
  numOfFails++;
}

int TestResult::failCount() const {
  return numOfFails;
}
 \end{c++}
\end{nodiff}

至此,测试通过。

\subsection{前置失败}

执行\ascii{TestCase::run}时,还未执行至\ascii{runTest}时,\ascii{setUp}可能提前失败了。新建一个失败的用例,模拟前置失败的场景。

\begin{nodiff}{test/mars/core/TestCaseSpec.cc}
 \begin{c++}
namespace {
  struct FailureOnSetUp : TestCase {
    bool wasRun = false;

  private:
    void setUp() override {
      throw AssertionError("product.cc:57", "expected value == 2, but got 3");
    }

    void runTest() override {
      wasRun = true;
    }
  };
}

TEST_F(TestCaseSpec, throw_assertion_error_on_setup) {
  FailureOnSetUp test;
  run(test);

  ASSERT_EQ(1, result.failCount());
  ASSERT_FALSE(test.wasRun);
}
 \end{c++}
\end{nodiff}

\subsubsection{捕获异常}

为了能够通过测试,当执行\ascii{TestCase::setUp}时,捕获异常并通知\ascii{TestResult},然后立即终止用例的执行。毕竟测试环境没准备好,强制执行\ascii{runTest}毫无意义。

\begin{leftbar}{src/mars/core/TestCase.cc}
 \begin{c++}
#include <mars/core/TestCase.h>
#include <mars/core/TestResult.h>
#include <mars/except/AssertionError.h>

void TestCase::runBare(TestResult& result) {
  bool succ = false;
  try {
    setUp();
    succ = true;
  } catch (const AssertionError&) {
    result.addFailure();
  }

  if (succ) {
    try {
      runTest();
    } catch (const AssertionError&) {
      result.addFailure();
    }
  }
  tearDown();
}
 \end{c++}
\end{leftbar}

至此,测试通过。

\subsubsection{消除重复}

显而易见,\ascii{setUp, runTest}的异常处理逻辑基本雷同,前者需要明确知道执行是否成功,而后者无所谓。观察\ascii{TestCase::setUp, TestCase::runTest}的共性,它们拥有相同的函数签名,可以使用指向成员函数的指针挖掘它们的共性。

\begin{nodiff}{相同的函数签名}
 \begin{c++}
using Method = void(TestCase::*)();
 \end{c++}
\end{nodiff}

提取公共函数\ascii{TestCase::protect},它返回\ascii{bool}类型。在头文件中声明私有函数\ascii{TestCase::protect}。

\begin{nodiff}{include/mars/core/TestCase.h}
 \begin{c++}
#include <mars/core/Test.h>
#include <mars/core/TestFixture.h>

struct TestCase : Test, private TestFixture {
  using Test::Test;

private:
  void run(TestResult&) override;
  int countTestCases() const override;

private:
  virtual void runTest() {}

private:
  using Method = void(TestCase::*)();
  bool protect(TestResult& result, Method method);
};
 \end{c++}
\end{nodiff}

在实现文件中,实现异常处理逻辑的代码复用。

\begin{nodiff}{src/mars/core/TestCase.cc}
 \begin{c++}
#include <mars/core/TestCase.h>
#include <mars/core/TestResult.h>
#include <mars/except/AssertionError.h>

bool TestCase::protect(TestResult& result, Method method) {
  bool succ = false;
  try {
    (this->*method)();
    succ = true;
  } catch (const AssertionError&) {
    result.addFailure();
  }
  return succ;
}

void TestCase::run(TestResult& result) {
  if (protect(result, &TestCase::setUp)) {
    protect(result, &TestCase::runTest);
  }
  tearDown();
}
 \end{c++}
\end{nodiff}

\subsection{后置失败}

\ascii{TestCase::run}执行时,\ascii{setUp, runTest}执行之后,无论成功与否,接着执行\ascii{tearDown}。但是,此时也存在失败的可能性。构建一个失败用例,模拟后置失败的场景。

\begin{nodiff}{test/mars/core/TestCaseSpec.cc}
 \begin{c++}
namespace {
  struct FailureOnTearDown : TestCase {
    void tearDown() override {
      throw AssertionError("product.cc:57", "expected value == 2, but got 3");
    }
  };
}

TEST_F(TestCaseSpec, throw_assertion_error_on_tear_down) {
  FailureOnTearDown test;
  run(test);
  ASSERT_EQ(1, result.failCount());
}
 \end{c++}
\end{nodiff}

\subsubsection{捕获异常}

当执行\ascii{TestCase::tearDown}时,捕获异常并通知\ascii{TestResult}。但是,无论\ascii{setUp, runTest}成败与否,\ascii{tearDown}都要强制执行,保证资源的安全释放。

\begin{nodiff}{src/mars/core/TestCase.cc}
 \begin{c++}
#include <mars/core/TestCase.h>
#include <mars/core/TestResult.h>
#include <mars/except/AssertionError.h>

bool TestCase::protect(TestResult& result, Method method) {
  bool succ = false;
  try {
    (this->*method)();
    succ = true;
  } catch (const AssertionError&) {
    result.addFailure();
  }
  return succ;
}

void TestCase::run(TestResult& result) {
  if (protect(result, &TestCase::setUp)) {
    protect(result, &TestCase::runTest);
  }
  protect(result, &TestCase::tearDown);
}
 \end{c++}
\end{nodiff}

至此,测试通过。

\section{错误}

\subsection{普通错误}

当抛出\ascii{AssertionError},称之为\ascii{Failure};当抛出其他异常,称之为\ascii{Error}。构建失败的用例,模拟抛出\ascii{std::exception}异常的场景。\ascii{std::exception}具有特殊性,它是标准库定义的异常类,也是所有标准异常的基类型。

\begin{nodiff}{test/mars/core/TestCaseSpec.cc}
 \begin{c++}
namespace {
  struct ErrorOnRunningTest : TestCase {
    void runTest() override {
      throw std::exception();
    }
  };
}

TEST_F(TestCaseSpec, throw_std_exception_on_run_test) {
  ErrorOnRunningTest test;
  run(test);

  ASSERT_EQ(0, result.failCount());
  ASSERT_EQ(1, result.errorCount());
}
 \end{c++}
\end{nodiff}

\subsubsection{通过编译}

为了能够通过编译,在\ascii{TestResult}中增加查询接口\ascii{errorCount}。

\begin{nodiff}{include/mars/core/TestResult.h}
 \begin{c++}
struct TestResult {
  TestResult();

  void addFailure();
  int failCount() const;

  int errorCount() const;

private:
  int numOfFails;
};
  \end{c++}
\end{nodiff}

\subsubsection{捕获异常}

在\ascii{TestCase::protect}异常处理逻辑中,捕获\ascii{std::exception}异常,并通过错误监听接口\ascii{addError}通知\ascii{TestResult},完成错误统计任务。

\begin{nodiff}{src/mars/core/TestCase.cc}
 \begin{c++}
#include <mars/core/TestCase.h>
#include <mars/core/TestResult.h>
#include <mars/except/AssertionError.h>

bool TestCase::protect(TestResult& result, Method method) {
  bool succ = false;
  try {
    (this->*method)();
    succ = true;
  } catch (const AssertionError&) {
    result.addFailure();
  } catch (const std::exception&) {
    result.addError();
  }
  return succ;
}

void TestCase::run(TestResult& result) {
  if (protect(result, &TestCase::setUp)) {
    protect(result, &TestCase::runTest);
  }
  protect(result, &TestCase::tearDown);
}
  \end{c++}
\end{nodiff}

\subsubsection{错误计数器}

在\ascii{TestResult}增加\ascii{numOfErrors}的错误计数器,及其错误监听接口\ascii{addError}。

\begin{nodiff}{include/mars/core/TestResult.h}
 \begin{c++}
struct TestResult {
  TestResult();

  void addFailure();
  int failCount() const;

  void addError();
  int errorCount() const;

private:
  int numOfFails;
  int numOfErrors;
};
  \end{c++}
\end{nodiff}

在实现文件中,实现错误计数器的初始化、累加,及其查询逻辑。

\begin{nodiff}{src/mars/core/TestResult.cc}
 \begin{c++}
#include <mars/core/TestResult.h>

TestResult::TestResult()
  : numOfFails(0), numOfErrors(0) {
}

void TestResult::addFailure() {
  numOfFails++;
}

int TestResult::failCount() const {
  return numOfFails;
}

void TestResult::addError() {
  numOfErrors++;
}

int TestResult::errorCount() const {
  return numOfErrors;
}
  \end{c++}
\end{nodiff}

\subsection{未知错误}

当抛出其他未知异常时,测试框架捕获并标记为\ascii{Error},提示用户被测系统存在严重漏洞。构造失败的用例,模拟此场景。

\begin{nodiff}{test/mars/TestCaseSpec.cc}
 \begin{c++}
namespace {
  struct NilException {};

  struct UnknownErrorOnRunningTest : TestCase {
  private:
    void runTest() override {
      throw NilException();
    }
  };
}

TEST_F(TestCaseSpec, throw_unknown_exception_on_running_test) {
  UnknownErrorOnRunningTest test;
  run(test);

  ASSERT_EQ(0, result.failCount());
  ASSERT_EQ(1, result.errorCount());
}
 \end{c++}
\end{nodiff}

\subsubsection{捕获异常}

在\ascii{TestCase::protected}异常处理逻辑中,使用通配符捕获所有未知异常,并通知\ascii{TestResult::addError}累加错误计数器。

\begin{nodiff}{src/mars/core/TestCase.cc}
 \begin{c++}
#include <mars/core/TestCase.h>
#include <mars/core/TestResult.h>
#include <mars/except/AssertionError.h>

bool TestCase::protect(TestResult& result, Method method) {
  bool succ = false;
  try {
    (this->*method)();
    succ = true;
  } catch (const AssertionError&) {
    result.addFailure();
  } catch (const std::exception&) {
    result.addError();
  } catch (...) {
    result.addError();
  }
  return succ;
}

void TestCase::run(TestResult& result) {
  if (protect(result, &TestCase::setUp)) {
    protect(result, &TestCase::runTest);
  }
  protect(result, &TestCase::tearDown);
}
 \end{c++}
\end{nodiff}

至此,测试通过。

\end{content}

