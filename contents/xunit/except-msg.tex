\begin{savequote}[45mm]
\ascii{Any fool can write code that a computer can understand. Good programmers write code that humans can understand.}
\qauthor{\ascii{- Martin Flower}}
\end{savequote}

\chapter{异常信息} 
\label{ch:except-msg}

\begin{content}

\end{content}

\section{异常信息}

\begin{content}

当用例失败后,用户期望终端打印如下信息,方便用户快速修复失败的用例。

\begin{leftbar}
 \begin{c++}
assertion fail in the runTest
/home/horance/code/cpp/cut/test/hamcrest/core/ComparatorTest.cpp:12
Expected: a value eq <254: int>
     but: <255: int> eq <254: int> got false
 \end{c++}
\end{leftbar}

异常信息包括如下几个重要逻辑信息:

\begin{enum}
  \eitem{\ascii{Why}: 指明用例失败的原因}
    \begin{enum}
      \eitem{\ascii{assertion fail}: 断言失败}
      \eitem{\ascii{uncaught std::exception}: 被测系统抛出\ascii{std::exception}或子异常}
      \eitem{\ascii{uncaught unknown exception}: 被测系统抛出其他未知异常}
    \end{enum}
  \eitem{\ascii{Where}: 异常抛出的地点}
    \begin{enum}
      \eitem{\ascii{in the setUp}}
      \eitem{\ascii{in the runTest}}
      \eitem{\ascii{in the tearDown}}
    \end{enum}
  \eitem{\ascii{What}: 异常携带的详细信息}
    \begin{enum}
      \eitem{\ascii{源文件}}
      \eitem{\ascii{行号}}
      \eitem{\ascii{异常详细信息}: \ascii{what}成员函数返回值}
    \end{enum}
\end{enum}

\end{content}

\section{断言失败}

\begin{content}

\subsection{测试用例}

重构既有的测试用例,当\ascii{runTest}因断言失败而抛出\ascii{AssertionError},\ascii{TestResult}缓存异常信息。

\begin{leftbar}
 \begin{c++}[caption={\ttfamily{test/mars/core/TestCaseSpec.cc}}]
namespace {
  struct AssertionFailedTest : TestCase {
    const char* expectMsg() const {
      return "assertion fail in the runTest\n"
              "TestCaseSpec.cpp:57\n"
              "expected value == 2, but got 3";
    }

  private:
    void runTest() override {
      throw AssertionError("TestCaseSpec.cpp:57", "expected value == 2, but got 3");
    }
  };
}

TEST_F(TestCaseSpec, throw_assertion_error_on_run_test) {
  AssertionFailedTest test;
  run(test);

  ASSERT_EQ(1, result.failCount());
  ASSERT_FALSE(result.getFailures().empty());
  ASSERT_EQ(test.expectMsg(), result.getFailures().front());
}
 \end{c++}
\end{leftbar}

\subsection{缓存异常}

\ascii{TestResult}建立相应的数据结构,缓存异常信息;同时,对外提供\ascii{getFailures}的查询接口,供外部用户获取异常信息。

\begin{leftbar}
 \begin{c++}[caption={\ttfamily{include/mars/core/TestResult.h}}]
#include <vector>
#include <string>

struct TestCaseFunctor;

struct TestResult {
  TestResult();

  int failCount() const;
  int errorCount() const;

  bool protect(const TestCaseFunctor&);

  const std::vector<std::string>& getFailures() const;

private:
  void onFail(std::string&& msg);
  void onError();

private:
  int numOfFails;
  int numOfErrors;

  std::vector<std::string> failures;
};
 \end{c++}
\end{leftbar}

当异常被捕获后,通过\ascii{onFail}监听接口添加至缓存。

\begin{leftbar}
 \begin{c++}[caption={\ttfamily{src/mars/core/TestResult.cc}}]
#include "mars/core/TestResult.h"
#include "mars/except/AssertionError.h"

// ...

const std::vector<std::string>& TestResult::getFailures() const {
  return failures;
}

bool TestResult::protect(const TestCaseFunctor& f) {
  try {
    return f();
  } catch (const AssertionError& e) {
    onFail(std::string("assertion fail") + " " + f.where() + "\n" + e.what());
  } catch (const std::exception& e) {
    onError();
  } catch (...) {
    onError();
  }
  return false;
}

void TestResult::onFail(std::string&& msg) {
  failures.emplace_back(std::move(msg));
  numOfFails++;
}
 \end{c++}
\end{leftbar}

\subsection{获取异常位置}

得益于\ascii{TestCaseFunctor}的可扩展性,可以扩展增加查询接口,可以方便携带额外的元数据。

\begin{leftbar}
 \begin{c++}[caption={\ttfamily{src/mars/core/TestResult.cc}}]
struct TestCaseFunctor {
  virtual ~TestCaseFunctor() {}
  virtual const char* where() const = 0;
  virtual bool operator()() const = 0;
};
 \end{c++}
\end{leftbar}

在\ascii{TestCase::run}实现中,通过修改\ascii{PROTECT}宏定义,并在匿名命名空间覆写\ascii{where}成员函数,报告异常发生的位置信息。

\begin{leftbar}
 \begin{c++}[caption={\ttfamily{src/mars/core/TestResult.cc}}]
namespace {
  struct Functor : TestCaseFunctor {
    using Method = void(TestCase::*)();

    Functor(TestCase* self, Method method, const char* place)
      : self(self), method(method), place(place) {
    }

  private:
    const char* where() const override {
      return place;
    }

    bool operator()() const override {
      (self->*method)();
      return true;
    }

  private:
    TestCase* self;
    Method method;
    const char* place;
  };
}

#define PROTECT(method) \
    result.protect(Functor(this, &TestCase::method, "in the "#method))

void TestCase::run(TestResult& result) {
  if (PROTECT(setUp)) {
    PROTECT(runTest);
  }
  PROTECT(tearDown);
}
 \end{c++}
\end{leftbar}

\end{content}

\section{错误:抛出std::exception}

\begin{content}

\subsection{测试用例}

模拟在\ascii{runTest}抛出错误,并断言异常信息。

\begin{leftbar}
 \begin{c++}[caption={\ttfamily{test/mars/core/TestCaseSpec.cc}}]
namespace {
  struct StdExceptionTest : TestCase {
    const char* expectMsg() const {
      return "uncaught std::exception in the runTest\n"
             "std::exception";
    }

  private:
    void runTest() override {
      throw std::exception();
    }
  };
}

TEST_F(TestCaseSpec, throw_std_exception_on_run_test) {
  StdExceptionTest test;
  run(test);

  ASSERT_EQ(0, result.failCount());
  ASSERT_EQ(1, result.errorCount());

  auto& errors = result.getErrors();
  ASSERT_FALSE(errors.empty());
  ASSERT_EQ(test.expectMsg(), errors.front());
}
 \end{c++}
\end{leftbar}

\subsection{缓存错误}

依葫芦画瓢,\ascii{TestResult}增加查询接口\ascii{getErrors},重构监听异常接口\ascii{onError}。

\begin{leftbar}
 \begin{c++}[caption={\ttfamily{include/mars/core/TestResult.h}}]
#include <vector>
#include <string>

struct TestCaseFunctor;

struct TestResult {
  TestResult();

  int failCount() const;
  int errorCount() const;

  bool protect(const TestCaseFunctor&);

  const std::vector<std::string>& getFailures() const;
  const std::vector<std::string>& getErrors() const;

private:
  void onFail(std::string&& msg);
  void onError(std::string&& msg);

private:
  int numOfFails;
  int numOfErrors;

  std::vector<std::string> failures;
  std::vector<std::string> errors;
};
 \end{c++}
\end{leftbar}

在实现文件中,与之前的实现雷同。

\begin{leftbar}
 \begin{c++}[caption={\ttfamily{src/mars/core/TestResult.cc}}]
#include "mars/core/TestResult.h"
#include "mars/core/internal/TestCaseFunctor.h"
#include "mars/except/AssertionError.h"

// ...

const std::vector<std::string>& TestResult::getFailures() const {
  return failures;
}

const std::vector<std::string>& TestResult::getErrors() const {
  return errors;
}

bool TestResult::protect(const TestCaseFunctor& f) {
  try {
    return f();
  } catch (const AssertionError& e) {
    onFail(std::string("assertion fail") + " " + f.where() + "\n" + e.what());
  } catch (const std::exception& e) {
    onError(std::string("uncaught std::exception") + " " + f.where() + "\n" + e.what());
  } catch (...) {
    onError("");
  }
  return false;
}

void TestResult::onFail(std::string&& msg) {
  failures.emplace_back(std::move(msg));
  numOfFails++;
}

void TestResult::onError(std::string&& msg) {
  errors.emplace_back(std::move(msg));
  numOfErrors++;
}
 \end{c++}
\end{leftbar}

\end{content}

\section{未知错误}

\begin{content}

遵循上述微步的循环,驱动实现未知错误的缓存特性。最终,\ascii{TestResult::protect}的异常处理逻辑如下所示。

\begin{leftbar}
 \begin{c++}[caption={\ttfamily{src/mars/core/TestResult.cc}}]
#include "mars/core/TestResult.h"
#include "mars/core/internal/TestCaseFunctor.h"
#include "mars/except/AssertionError.h"

// ...

const std::vector<std::string>& TestResult::getFailures() const {
  return failures;
}

const std::vector<std::string>& TestResult::getErrors() const {
  return errors;
}

bool TestResult::protect(const TestCaseFunctor& f) {
  try {
    return f();
  } catch (const AssertionError& e) {
    onFail(std::string("assertion fail") + " " + f.where() + "\n" + e.what());
  } catch (const std::exception& e) {
    onError(std::string("uncaught std::exception") + " " + f.where() + "\n" + e.what());
  } catch (...) {
    onError(std::string("uncaught unknown exception") + " " + f.where() + "\n" + "");
  }
  return false;
}

void TestResult::onFail(std::string&& msg) {
  failures.emplace_back(std::move(msg));
  numOfFails++;
}

void TestResult::onError(std::string&& msg) {
  errors.emplace_back(std::move(msg));
  numOfErrors++;
}
 \end{c++}
\end{leftbar}

\end{content}

\section{值对象}

\begin{content}

\subsection{测试用例}

使用字符串表示异常信息是一种脆弱的设计,可以将其重构为值对象,从而得到更稳定的设计。首先,重构既有的测试用例,使其失败。

\begin{leftbar}
 \begin{c++}[caption={\ttfamily{test/mars/core/TestCaseSpec.cc}}]
namespace {
  struct AssertionFailedTest : TestCase {
    const char* expectMsg() const {
      return "assertion fail in the runTest\n"
             "test.cpp:57\n"
             "expected value == 2, but got 3";
    }

  private:
    void runTest() override {
      throw AssertionError("test.cpp:57", "expected value == 2, but got 3");
    }
  };
}

TEST_F(TestCaseSpec, cache_failure_if_throw_assertion_error_on_run_test) {
  AssertionFailedTest test;
  run(test);

  auto& failures = result.getFailures();
  ASSERT_FALSE(failures.empty());

  auto failure = failures.front();
  ASSERT_TRUE(failure->isFailure());
  ASSERT_EQ(test.expectMsg(), failure->getExceptionMsg());
}
 \end{c++}
\end{leftbar}

测试用例使用\ascii{auto}自动类型推演,没有显式给出具体的类型。但是,该值对象约定具有\ascii{isFailure, getExceptionMsg}的接口定义。另外,为了避免拷贝值对象损失性能,这里使用指针类型。

\subsection{定义值对象}

\ascii{TestFailure}是一个值对象,相对原生的\ascii{std::string},它相对更加稳定和抽象,并更具有表达力。

\begin{leftbar}
 \begin{c++}[caption={\ttfamily{include/mars/except/TestFailure.h}}]
#include <string>

struct TestFailure {
  TestFailure(std::string&& msg, bool failure);

  bool isFailure() const;
  const std::string& getExceptionMsg() const;

private:
  std::string msg;
  bool failure;
};
 \end{c++}
\end{leftbar}

其实现非常简单。

\begin{leftbar}
 \begin{c++}[caption={\ttfamily{src/mars/except/TestFailure.cc}}]
#include "mars/except/TestFailure.h"

TestFailure::TestFailure(std::string&& msg, bool failure)
  : msg(std::move(msg)), failure(failure) {
}

bool TestFailure::isFailure() const {
  return failure;
}

const std::string& TestFailure::getExceptionMsg() const {
  return msg;
}
 \end{c++}
\end{leftbar}

\subsection{使用值对象}

为了避免拷贝\ascii{TestFailure},在\ascii{TestResult}存储异常列表时使用动态内存。

\begin{leftbar}
 \begin{c++}[caption={\ttfamily{include/mars/core/TestResult.h}}]
#include <vector>
#include <string>
#include "mars/except/TestFailure.h"

struct TestCaseFunctor;

struct TestResult {
  TestResult();
  ~TestResult();

  int failCount() const;
  int errorCount() const;

  bool protect(const TestCaseFunctor&);

  const std::vector<TestFailure*>& getFailures() const;
  const std::vector<std::string>& getErrors() const;

private:
  void onFail(std::string&& msg);
  void onError(std::string&& msg);

private:
  int numOfFails;
  int numOfErrors;

  std::vector<TestFailure*> failures;
  std::vector<std::string> errors;
};
 \end{c++}
\end{leftbar}

实现文件中,主要完成\ascii{TestFailure}的构造,及其动态内存管理。

\begin{leftbar}
 \begin{c++}[caption={\ttfamily{src/mars/core/TestResult.cc}}]
#include "mars/core/TestResult.h"
#include "mars/core/internal/TestCaseFunctor.h"
#include "mars/except/AssertionError.h"

// ...

TestResult::~TestResult() {
  for (auto f : failures) {
    delete f;
  }
}

const std::vector<TestFailure*>& TestResult::getFailures() const {
  return failures;
}

const std::vector<std::string>& TestResult::getErrors() const {
  return errors;
}

bool TestResult::protect(const TestCaseFunctor& f) {
  try {
    return f();
  } catch (const AssertionError& e) {
    onFail(std::string("assertion fail") + " " + f.where() + "\n" + e.what());
  } catch (const std::exception& e) {
    onError(std::string("uncaught std::exception") + " " + f.where() + "\n" + e.what());
  } catch (...) {
    onError(std::string("uncaught unknown exception") + " " + f.where() + "\n" + "");
  }
  return false;
}

void TestResult::onFail(std::string&& msg) {
  failures.push_back(new TestFailure(std::move(msg), true));
  numOfFails++;
}

void TestResult::onError(std::string&& msg) {
  errors.emplace_back(std::move(msg));
  numOfErrors++;
}
 \end{c++}
\end{leftbar}

至此,测试通过。

\subsection{实现统一}

为了使用\ascii{TestFailure}统一描述失败与错误两种场景,重构既有用例。

\begin{leftbar}
 \begin{c++}[caption={\ttfamily{test/mars/core/TestCaseSpec.cc}}]
namespace {
  struct StdExceptionTest : TestCase {
    const char* expectMsg() const {
      return "uncaught std::exception in the runTest\n"
              "std::exception";
    }

  private:
    void runTest() override {
      throw std::exception();
    }
  };
}

TEST_F(TestCaseSpec, cache_error_if_throw_std_exception_on_run_test) {
  StdExceptionTest test;
  run(test);

  auto& errors = result.getFailures();
  ASSERT_FALSE(errors.empty());

  auto error = errors.front();
  ASSERT_TRUE(error->isError());
  ASSERT_EQ(test.expectMsg(), error->getExceptionMsg());
}
 \end{c++}
\end{leftbar}


为了能够通过测试用例,新增\ascii{TestFailure::isError}接口。

\begin{leftbar}
 \begin{c++}[caption={\ttfamily{src/mars/except/TestFailure.cc}}]
bool TestFailure::isError() const {
  return !isFailure();
}

bool TestFailure::isFailure() const {
  return failure;
}
 \end{c++}
\end{leftbar}

在\ascii{TestResult}中便可以删除既有的\ascii{errors}列表,及其相关的查询接口\ascii{getErrors};统一使用\ascii{failures}一个缓存列表,及其使用\ascii{getFailures}一个查询接口。至此,消除了失败与错误两种重复表示,统一了接口的使用方式。

\begin{leftbar}
 \begin{c++}[caption={\ttfamily{include/mars/core/TestResult.h}}]
#include <string>
#include <vector>

struct TestFailure;
struct TestCaseFunctor;

struct TestResult {
  TestResult();
  ~TestResult();

  int failCount() const;
  int errorCount() const;

  bool protect(const TestCaseFunctor&);

  const std::vector<TestFailure*>& getFailures() const;

private:
  void onFail(std::string&& msg);
  void onError(std::string&& msg);

private:
  int numOfFails;
  int numOfErrors;

  std::vector<TestFailure*> failures;
};
 \end{c++}
\end{leftbar}


\begin{leftbar}
 \begin{c++}[caption={\ttfamily{src/mars/core/TestResult.cc}}]
#include "mars/core/TestResult.h"
#include "mars/core/internal/TestCaseFunctor.h"
#include "mars/except/AssertionError.h"
#include "mars/except/TestFailure.h"

// ...

TestResult::~TestResult() {
  for (auto f : failures) {
    delete f;
  }
}

const std::vector<TestFailure*>& TestResult::getFailures() const {
  return failures;
}

bool TestResult::protect(const TestCaseFunctor& f) {
  try {
    return f();
  } catch (const AssertionError& e) {
    onFail(std::string("assertion fail") + " " + f.where() + "\n" + e.what());
  } catch (const std::exception& e) {
    onError(std::string("uncaught std::exception") + " " + f.where() + "\n" + e.what());
  } catch (...) {
    onError(std::string("uncaught unknown exception") + " " + f.where() + "\n" + "");
  }
  return false;
}

void TestResult::onFail(std::string&& msg) {
  failures.push_back(new TestFailure(std::move(msg), true));
  numOfFails++;
}

void TestResult::onError(std::string&& msg) {
  failures.push_back(new TestFailure(std::move(msg), false));
  numOfErrors++;
}
 \end{c++}
\end{leftbar}

\subsection{工厂方法}

提取两个创建\ascii{TestFailure}的工厂方法\ascii{failure, error},及其创建异常信息的工厂方法\ascii{msg},得到可复用的代码实现。

\begin{leftbar}
 \begin{c++}[caption={\ttfamily{src/mars/core/TestResult.cc}}]
namespace {
  std::string msg(const char* why, const char* where, const char* what = "") {
    return std::string(why) + " " + where + "\n" + what;
  }

  TestFailure* failure(std::string&& msg) {
    return new TestFailure(std::move(msg), true);
  }

  TestFailure* error(std::string&& msg) {
    return new TestFailure(std::move(msg), false);
  }
}

bool TestResult::protect(const TestCaseFunctor& f) {
  try {
    return f();
  } catch (const AssertionError& e) {
    onFail(failure(msg("assertion fail", f.where(), e.what())));
  } catch (const std::exception& e) {
    onFail(error(msg("uncaught std::exception", f.where(), e.what())));
  } catch (...) {
    onFail(error(msg("uncaught unknown exception", f.where())));
  }
  return false;
}

void TestResult::onFail(TestFailure* f) {
  failures.push_back(f);
  f->isFailure() ? numOfFails++ : numOfErrors++; 
}
 \end{c++}
\end{leftbar}

\end{content}