\begin{savequote}[45mm]
\ascii{Any fool can write code that a computer can understand. Good programmers write code that humans can understand.}
\qauthor{\ascii{- Martin Flower}}
\end{savequote}

\chapter{自动发现} 
\label{ch:auto-register}

\begin{content}

\end{content}

% 太古代、元古代、古生代、中生代和新生代.

\section{太古代}

\begin{content}

\subsection{雏形}

基于目前\ascii{xUnit Mars}的框架实现,已经具备条件设计\ascii{xUnit Mars}风格的测试用例了。目前,\ascii{xUnit Mars}缺失\ascii{DSL}的构建能力,因此测试用例遵循\cpp{}函数命名的基本规范。根据场景的实际需求,覆写相关的成员函数\ascii{setUp, tearDown}。

另外,\ascii{xUnit Mars}目前还未支持\emph{断言接口},暂且使用\ascii{Google Test}提供的基本特性完成。例如,设计实现了\ascii{StackSpec}的测试用例套件。

\begin{nodiff}{test/mars/auto/ManualStackSpec.cc}
 \begin{c++}
#include <gtest/gtest.h>
#include <mars/core/TestFixture.h>
#include <stack>

namespace {
  struct StackSpec : TestFixture {
    std::stack<int> v;

    void setUp() override {
      v.push(1);
      v.push(2);
    }

    void apply_pop_0_time() {
      ASSERT_EQ(2, v.top());
    }

    void apply_pop_1_time() {
      v.pop();
      ASSERT_EQ(1, v.top());
    }

    void apply_pop_2_times() {
      v.pop();
      v.pop();
      ASSERT_TRUE(v.empty());
    }
  };
}
 \end{c++}
\end{nodiff}

\subsection{手动注册}

目前,\ascii{xUnit Mars}缺失用例\emph{自动注册}、\emph{自动发现}的基本功能。此时,为了能够运行\ascii{StackSpec}中的测试方法,暂且通过手动注册完成。

\begin{nodiff}{test/mars/auto/ManualStackSpec.cc}
 \begin{c++}
#include <gtest/gtest.h>
#include <mars/core/TestSuite.h>
#include <mars/core/TestMethod.h>

// ...

TEST(ManualStackSpec, run_stack_spec) {
  TestSuite suite("StackSpec");
  suite.add(new TestMethod<StackSpec>(&StackSpec::apply_pop_0_time));
  suite.add(new TestMethod<StackSpec>(&StackSpec::apply_pop_1_time));
  suite.add(new TestMethod<StackSpec>(&StackSpec::apply_pop_2_times));
}
 \end{c++}
\end{nodiff}

\subsection{手动执行}

目前,\ascii{xUnit Mars}缺失测试用例\emph{自动运行}的基本功能,需要手动搭建测试用例的运行环境,并手动启动的执行引擎。

搭建运行环境,主要包括创建\ascii{TestResult}对象,配置一个或多个监听\ascii{TestResult}变化的\ascii{TestListener}实例。启动执行引擎,则通过调用\ascii{TestResult::runRooTest}接口触发。例如,此处仅提供一个监听器\ascii{TextProgress},用于监测\ascii{xUnit Mars}内部运行是否符合预期行为。

\begin{nodiff}{test/mars/auto/ManualStackSpec.cc}
 \begin{c++}
#include <gtest/gtest.h>
#include <mars/core/TestResult.h>
#include <mars/listener/text/TextProgress.h>
#include <sstream>

namespace {
  struct ManualStackSpec : testing::Test {
  protected:
    ManualStackSpec() : progress(ss) {
      result.addListener(progress);
    }

  protected:
    std::ostringstream ss;
    TestResult result;
    TextProgress progress;
  };
}

TEST_F(ManualStackSpec, run_stack_spec) {
  TestSuite suite("StackSpec");
  suite.add(new TestMethod<StackSpec>(&StackSpec::apply_pop_0_time));
  suite.add(new TestMethod<StackSpec>(&StackSpec::apply_pop_1_time));
  suite.add(new TestMethod<StackSpec>(&StackSpec::apply_pop_2_times));

  result.runRootTest(suite);
  ASSERT_EQ(ss.str(), "starting...\n***\nend.\n");
}
 \end{c++}
\end{nodiff}

至此,测试通过。

\end{content}

\section{元古代}

\begin{content}

\subsection{自动化}

手动注册\ascii{TestMethod}非常笨拙,不仅容易出错,而且可读性极差。可以引入实用的宏定义,改善用户体验。但是,其并不能完美解决核心问题,用户依然需要手动注册。

因此,需要考虑引入自动化的实现机制,使得用户彻底不用关心这件繁琐的事情。重构既有的测试用例,并使其失败。首当其冲,将\ascii{ManualStackSpec}重命名为\ascii{AutoStackSpec}。

在\ascii{AutoStackSpec}中,创建用例树的根节点\ascii{root},并使用静态工厂方法\ascii{suite}完成初始化。在\ascii{suite}实现中,首先试图获取\ascii{TestFactory}工厂实例,然后创建\ascii{StackSpec}的所有测试方法。

最后,核心问题转变为:如何自动发现\ascii{StackSpec}之中的所有测试方法,并将它们的原型信息自动注册到\ascii{TestFixtureFactory<StackSpec>}单例之中?

重构既有的测试用例,使其失败。删除所有手动注册\ascii{TestMethod}实例的实现,通过调用工厂方法自动创建\ascii{StackSpec}对应的\ascii{TestMethod}实例。

\begin{nodiff}{test/mars/auto/AutoStackSpec.cc}
 \begin{c++}
#include <gtest/gtest.h>
#include <mars/core/TestResult.h>
#include <mars/listener/text/TextProgress.h>
#include <mars/factory/TestFixtureFactory.h>
#include <sstream>

// ...

namespace {
  struct AutoStackSpec : testing::Test {
  protected:
    AutoStackSpec() : progress(ss), root(suite()) {
      result.addListener(progress);
    }

    ~AutoStackSpec() {
      delete root;
    }

  private:
    static ::Test* suite() {
      return TestFixtureFactory<StackSpec>::instance().make();
    }

  protected:
    std::ostringstream ss;
    TestResult result;
    TextProgress progress;
    ::Test* root;
  };
}

TEST_F(AutoStackSpec, run_stack_spec) {
  result.runRootTest(*root);
  ASSERT_EQ(ss.str(), "starting...\n***\nend.\n");
}
 \end{c++}
\end{nodiff}

\subsubsection{空仓库}

为了能够通过编译,创建单例的工厂类\ascii{TestFixtureFactory},其工厂方法暂且返回一个空的\ascii{TestSuite}实例。

\begin{nodiff}{include/mars/factory/TestFixtureFactory.h}
 \begin{c++}
#include <mars/util/Singleton.h>
#include <mars/core/TestSuite.h>

GENERIC_SINGLETON(TestFixtureFactory, Fixture) {
  Test* make() {
    return new TestSuite;
  }
};
 \end{c++}
\end{nodiff}

\ascii{TestFixtureFactory}是一个单例,但与普通单例实现存在微妙的差异。它拥有一个泛型参数\ascii{Fixture},使得每个\ascii{Fixture}类型,都会存在一个唯一的、独立的、隔离的存储空间。例如,在\ascii{StackSpec.cc}中,将自动创建一个\ascii{TestFixtureFactory<StackaSpec>}的单例,与其他\ascii{TestFixtureFactory<Fixture>}实例相隔离。

\subsubsection{实用宏:单例}

实现单例遵循固有的实现模式,为了消除\ascii{Singleton}的重复设计,提取\ascii{Singleton}的公共实现,并放在产品实现的基础设施之中。引入\ascii{Singleton}将免费得到\ascii{instance}的公共实现。切忌在每个单例实现中,都复制拷贝一份单例实现;否则将引入严重的重复设计。

为了方便用户快速创建单例类,此处提供两个实用的宏定义。其中,\ascii{SINGLETON}用于创建普通的单例,而\ascii{GENERIC\_SINGLETON}创建带有泛型参数的单例。

\begin{nodiff}{include/mars/util/Singleton.h}
 \begin{c++}
template<typename T>
struct Singleton {
  static T& instance() {
    static T inst;
    return inst;
  }

protected:
  Singleton() {}
};

#define SINGLETON(type) \
  struct type : Singleton<type>

#define GENERIC_SINGLETON(type, ...)         \
  template<typename __VA_ARGS__>             \
  struct type : Singleton<type<__VA_ARGS__>>
 \end{c++}
\end{nodiff}

至此,仅能通过编译,但不能通过测试。

\subsection{代码注入}

为了能够通过测试,必须提供一种自动注册的实现机制,萃取\ascii{StackSpec}测试方法的原型信息,并自动地注册到\ascii{TestFixtureFactory<StackSpec>}单例之中。但是,由于\cpp{}缺失反射的机制,发现测试方法变得相当棘手。

应用\emph{代码注入}技术,当生成\ascii{StackSpec}实例时,将自动执行这些注入的代码,完成\ascii{StackSpec}测试方法原型的萃取过程。

\begin{nodiff}{test/mars/auto/StackSpec.cc}
 \begin{c++}
#include <gtest/gtest.h>
#include <mars/core/TestFixture.h>
#include <mars/auto/AutoTestMethod.h>
#include <stack>

namespace {
  struct StackSpec : TestFixture {
    std::stack<int> v;

    void setUp() override {
      v.push(1);
      v.push(2);
    }

  private:
    AutoTestMethod m1{1, "apply_pop_0_time", &StackSpec::apply_pop_0_time};

  public:
    void apply_pop_0_time() {
      ASSERT_EQ(2, v.top());
    }

  private:
    AutoTestMethod m2{2, "apply_pop_1_time", &StackSpec::apply_pop_1_time};

  public:
    void apply_pop_1_time() {
      v.pop();
      ASSERT_EQ(1, v.top());
    }

  private:
    AutoTestMethod m3{3, "apply_pop_2_times", &StackSpec::apply_pop_2_times};

  public:
    void apply_pop_2_times() {
      v.pop();
      v.pop();
      ASSERT_TRUE(v.empty());
    }
  };
}
 \end{c++}
\end{nodiff}

\subsubsection{自动注册}

为了能够通过编译,创建\ascii{AutoTestMethod}。在构造函数中,自动地注册测试方法的原型信息至\ascii{TestFixtureFactory<Fixture>}单例之中。

\begin{nodiff}{include/mars/auto/AutoTestMethod.h}
 \begin{c++}
#include <mars/factory/TestFixtureFactory.h>
#include <mars/core/Method.h>

struct AutoTestMethod {
  template <typename Fixture>
  AutoTestMethod(int id, const char* name, Method<Fixture> method) {
    auto& registry = TestFixtureFactory<Fixture>::instance();
    registry.add(id, name, method);
  }
};
 \end{c++}
\end{nodiff}

\subsubsection{注册接口}

为了能够通过编译,暂且空实现注册接口\ascii{TestFixtureFactory::add}。同时,重构工厂方法\ascii{TestFixtureFactory::make}的实现逻辑,当它被第一次调用时,初始创建一个局部的、静态的\ascii{StackSpec}实例,此刻将触发创建和初始化\ascii{StackSpec}实例相关的所有\ascii{AutoTestMethod}实例,最终在\ascii{AutoTestMethod}的构造函数中完成对应测试方法原型的萃取过程。

因为这个过程相当隐晦,目前也没有更好的实现方法替换。因此,补充相关注释信息至关重要。众所周知,\ascii{StackSpec}实例会被经常创建和销毁。因此,\ascii{TestFixtureFactory}的实现必须提供一种预先的、轻量级的准入机制,禁止重复,或覆盖行为发生。

\begin{nodiff}{include/mars/auto/TestMethodRegistry.h}
 \begin{c++}
#include <mars/util/Singleton.h>
#include <mars/core/TestSuite.h>
#include <mars/core/Method.h>

GENERIC_SINGLETON(TestFixtureFactory, Fixture) {
  void add(int id, const char* name, Method<Fixture> method) {
  }

  Test* make() {
    static Fixture dummy; // register all test methods to this.
    return new TestSuite;
  }
};
 \end{c++}
\end{nodiff}

至此,编译通过,依然未能通过测试。

\subsubsection{原子工厂}

为了能够存储测试方法的原型信息,封装\ascii{<name, method>}的二元组信息,并将其命名为\ascii{TestMethodFactory},它根据测试方法的原型信息创建\ascii{TestMethod}实例。

\begin{nodiff}{include/mars/factory/TestMethodFactory.h}
 \begin{c++}
#include <mars/core/TestMethod.h>

template <typename Fixture>
struct TestMethodFactory {
  TestMethodFactory(const char* name, Method<Fixture> method)
    : name(name), method(method) {}

  Test* make() {
    return new TestMethod<Fixture>(method, name);
  }

private:
  const char* name;
  Method<Fixture> method;
};
 \end{c++}
\end{nodiff}

\subsubsection{实现注册接口}

在\ascii{TestFixtureFactory}实现中,创建类型为\ascii{std::unordered\_map}的存储结构,用于存储和检索\ascii{TestMethodFactory}实例。在\ascii{TestMethodRegistry::add}实现中,预先判断\ascii{id}对应的原子工厂是否已经存在。如果存在,避免重复注册;否则,在\ascii{map}中创建相应的记录。

\begin{nodiff}{include/mars/auto/TestMethodRegistry.h}
 \begin{c++}
#include <mars/util/Singleton.h>
#include <mars/core/TestSuite.h>
#include <mars/factory/TestMethodFactory.h>
#include <unordered_map>

GENERIC_SINGLETON(TestFixtureFactory, Fixture) {
  void add(int id, const char* name, Method<Fixture> method) {
    if (!exist(id)) {
      registry.insert({id, {name, method}});
    }
  }

  Test* make() {
    static Fixture dummy; // register all test methods to this.
    return new TestSuite;
  }

private:
  bool exist(int id) const {
    return registry.find(id) != registry.end();
  }

private:
  std::unordered_map<int, TestMethodFactory<Fixture>> registry;
};
 \end{c++}
\end{nodiff}

\subsubsection{实现工厂方法}

提取工厂方法\ascii{suite},遍历已注册的\ascii{TestMethodFactory}实例,创建对应\ascii{TestMethod}实例。最终,将所有\ascii{TestMethod}实例打包为一个\ascii{TestSuite}实例,返回给客户。

\begin{nodiff}{include/mars/factory/TestFixtureFactory.h}
 \begin{c++}
#include <mars/util/Singleton.h>
#include <mars/core/TestSuite.h>
#include <mars/factory/TestMethodFactory.h>
#include <unordered_map>

GENERIC_SINGLETON(TestFixtureFactory, Fixture) {
  void add(int id, const char* name, Method<Fixture> method) {
    if (!exist(id)) {
      registry.insert({id, {name, method}});
    }
  }

  Test* make() {
    static Fixture dummy; // register all test methods to this.
    return suite();
  }

private:
  bool exist(int id) const {
    return registry.find(id) != registry.end();
  }

  Test* suite() {
    auto result = new TestSuite;
    for (auto& i : registry) {
      result->add(i.second.make());
    }
    return result;
  }

private:
  std::unordered_map<int, TestMethodFactory<Fixture>> registry;
};
 \end{c++}
\end{nodiff}

此时,测试通过。

\subsection{提炼工厂}

基于上述观察,提取关键抽象\ascii{TestFactory}。

\begin{nodiff}{include/mars/auto/TestFactory.h}
 \begin{c++}
struct Test;

struct TestFactory {
  virtual Test* make() = 0;
  virtual ~TestFactory() {}
};
 \end{c++}
\end{nodiff}

\subsubsection{原子工厂}

\begin{nodiff}{include/mars/factory/TestMethodFactory.h}
 \begin{c++}
#include <mars/factory/TestFactory.h>
#include <mars/core/TestMethod.h>

template <typename Fixture>
struct TestMethodFactory : TestFactory {
  TestMethodFactory(const char* name, Method<Fixture> method)
    : name(name), method(method) {}

private:
  Test* make() override {
    return new TestMethod<Fixture>(method, name);
  }

private:
  const char* name;
  Method<Fixture> method;
};
 \end{c++}
\end{nodiff}

\subsubsection{复合工厂}

\begin{nodiff}{include/mars/factory/TestSuiteFactory.h}
 \begin{c++}
#include <mars/factory/TestFactory.h>
#include <vector>

struct TestSuiteFactory : TestFactory {
  void add(TestFactory&);

protected:
  Test* make() override;

private:
  std::vector<TestFactory*> factories;
};
 \end{c++}
\end{nodiff}

\begin{nodiff}{src/mars/auto/TestFactorySuite.cc}
 \begin{c++}
#include <mars/factory/TestSuiteFactory.h>
#include <mars/core/TestSuite.h>

void TestSuiteFactory::add(TestFactory& f) {
  factories.push_back(&f);
}

Test* TestSuiteFactory::make() {
  auto suite = new TestSuite;
  for (auto f : factories) {
    suite->add(f->make());
  }
  return suite;
}
 \end{c++}
\end{nodiff}

\subsubsection{搬迁基类}

\begin{nodiff}{include/mars/factory/TestFixtureFactory.h}
 \begin{c++}
#include <mars/factory/TestMethodFactory.h>
#include <mars/factory/TestSuiteFactory.h>
#include <mars/util/Singleton.h>
#include <unordered_map>

GENERIC_SINGLETON(TestFixtureFactory, Fixture) EXTENDS(TestSuiteFactory) {
  void add(int id, const char* name, Method<Fixture> method) {
    if (!exist(id)) {
      registry.insert({id, {name, method}});
      TestSuiteFactory::add(registry.at(id));
    }
  }

private:
  Test* make() override {
    static Fixture dummy; // register all test methods to registry.
    return TestSuiteFactory::make();
  }

private:
  bool exist(int id) const {
    return registry.find(id) != registry.end();
  }

private:
  std::unordered_map<int, TestMethodFactory<Fixture>> registry;
};
 \end{c++}
\end{nodiff}

通过测试。

\subsection{执行器}

当增加另外一组测试用例套件时,例如\ascii{QueueSpec}。

\begin{nodiff}{test/mars/auto/AutoQueueSpec.cc}
 \begin{c++}
#include <mars/core/TestFixture.h>
#include <mars/auto/AutoTestMethod.h>
#include <queue>

namespace {
  struct QueueSpec : TestFixture {
    std::queue<int> q;

    void setUp() override {
      q.push(1);
      q.push(2);
    }

    void apply_pop_0_time() {
      ASSERT_EQ(1, q.front());
      ASSERT_EQ(2, q.back());
    }

    AutoTestMethod m1{1, "apply_pop_0_time", &QueueSpec::apply_pop_0_time};

    void apply_pop_1_time() {
      q.pop();
      ASSERT_EQ(2, q.front());
      ASSERT_EQ(2, q.back());
    }

    AutoTestMethod m2{2, "apply_pop_1_time", &QueueSpec::apply_pop_1_time};

    void apply_pop_2_times() {
      q.pop();
      q.pop();
      ASSERT_TRUE(q.empty());
    }

    AutoTestMethod m3{3, "apply_pop_2_times", &QueueSpec::apply_pop_2_times};
  };
}
 \end{c++}
\end{nodiff}

\subsubsection{手动执行}

\begin{nodiff}{test/mars/auto/AutoQueueSpec.cc}
 \begin{c++}
#include <gtest/gtest.h>
#include <mars/core/TestResult.h>
#include <mars/listener/text/TextProgress.h>
#include <mars/factory/TestFixtureFactory.h>
#include <sstream>

namespace {
  struct AutoQueueSpec : testing::Test {
  protected:
    AutoQueueSpec() : progress(ss), root(suite()) {
      result.addListener(progress);
    }

    ~AutoQueueSpec() {
      delete root;
    }

  private:
    static ::Test* suite() {
      TestFactory& factory = TestFixtureFactory<QueueSpec>::instance();
      return factory.make();
    }

  protected:
    std::ostringstream ss;
    TestResult result;
    TextProgress progress;
    ::Test* root;
  };
}

TEST_F(AutoQueueSpec, run_queue_spec) {
  result.runRootTest(*root);
  ASSERT_EQ("starting...\n***\nend.\n", ss.str());
}
 \end{c++}
\end{nodiff}

\subsubsection{消除重复}

\begin{nodiff}{test/mars/auto/AutoSpec.cc}
 \begin{c++}
#include <mars/runner/TestRunner.h>
#include <mars/listener/text/TextProgress.h>
#include <mars/factory/TestFixtureFactory.h>
#include <stack>
#include <queue>
#include <sstream>

// ...

namespace {
  template <typename Fixture>
  struct GenericAutoSpec : testing::Test {
  protected:
    GenericAutoSpec()
      : progress(ss)
      , runner(progress, factory()) {}

  private:
    static TestFactory& factory() {
      return TestFixtureFactory<Fixture>::instance();
    }

  protected:
    std::ostringstream ss;
    TextProgress progress;
    TestRunner runner;
  };

  using AutoStackSpec = GenericAutoSpec<StackSpec>;
  using AutoQueueSpec = GenericAutoSpec<QueueSpec>;
}

TEST_F(AutoStackSpec, run_stack_spec) {
  runner.run();
  ASSERT_EQ("starting...\n***\nend.\n", ss.str());
}

TEST_F(AutoQueueSpec, run_queue_spec) {
  runner.run();
  ASSERT_EQ("starting...\n***\nend.\n", ss.str());
}
 \end{c++}
\end{nodiff}

\subsubsection{实现执行器}

\begin{nodiff}{include/mars/runner/TestRunner.h}
 \begin{c++}
#include <mars/core/TestResult.h>

struct TestFactory;

struct TestRunner {
  TestRunner(TestListener&, TestFactory&);
  ~TestRunner();

  void run();

private:
  TestResult result;
  Test* root;
};
 \end{c++}
\end{nodiff}

\begin{nodiff}{src/mars/runner/TestRunner.cc}
 \begin{c++}
#include <mars/runner/TestRunner.h>
#include <mars/factory/TestFactory.h>
#include <mars/core/Test.h>

TestRunner::TestRunner(TestListener& listener, TestFactory& factory)
  : root(factory.make()) {
  result.addListener(listener);
}

TestRunner::~TestRunner() {
  delete root;
}

void TestRunner::run() {
  result.runRootTest(*root);
}
 \end{c++}
\end{nodiff}

\end{content}

\section{古生代}

\begin{content}

\subsection{自动化}

\begin{nodiff}{include/mars/startup/Startup.h}
 \begin{c++}
void startup(int argv, char** argc);
 \end{c++}
\end{nodiff}

\begin{nodiff}{src/mars/startup/Startup.cc}
 \begin{c++}
#include <mars/startup/Startup.h>
#include <mars/listener/text/ColorfulPrinter.h>
#include <mars/factory/TestSuiteFactory.h>
#include <mars/runner/TestRunner.h>

void startup(int, char**) {
  ColorfulPrinter printer;
  TestRunner runner(printer, TestSuiteFactory::root());
  runner.run();
}
 \end{c++}
\end{nodiff}

\subsubsection{main函数}

\begin{nodiff}{src/mars/startup/Startup.cc}
 \begin{c++}
#include <gtest/gtest.h>
#include <mars/startup/Startup.h>

int main(int argc, char** argv) {
  testing::InitGoogleTest(&argc, argv);
  startup(argc, argv);
  return RUN_ALL_TESTS();
}
 \end{c++}
\end{nodiff}

\subsubsection{根节点}

\begin{nodiff}{src/mars/factory/TestSuiteFactory.cc}
 \begin{c++}
#include <mars/factory/TestSuiteFactory.h>
#include <mars/core/TestSuite.h>

// ...

TestSuiteFactory& TestSuiteFactory::root() {
  static TestSuiteFactory inst;
  return inst;
}
 \end{c++}
\end{nodiff}

\subsection{代码注入}

\begin{nodiff}{test/mars/auto/AutoSpec.cc}
 \begin{c++}
#include <mars/core/TestFixture.h>
#include <mars/auto/AutoTestFixture.h>
#include <stack>
#include <queue>

namespace {
  struct StackSpec;
  static AutoTestFixture<StackSpec> suite1;

  struct StackSpec : TestFixture {
    // ...
  };

  struct QueueSpec;
  static AutoTestFixture<QueueSpec> suite2;

  struct QueueSpec : TestFixture {
    // ...
  };  
}
 \end{c++}
\end{nodiff}

\subsubsection{自动注册}

\ascii{AutoTestFixture}在构造时,将每个\ascii{TestFixtureFactory<Fixture>}单例自动地注册到根节点\ascii{TestSuiteFactory::root}单例之中。

\begin{nodiff}{include/mars/auto/AutoTestSuite.h}
 \begin{c++}
#include <mars/factory/TestFixtureFactory.h>
#include <mars/factory/TestSuiteFactory.h>

template<typename Fixture>
struct AutoTestFixture {
  AutoTestFixture(const char* name) {
    auto& root = TestSuiteFactory::root();
    auto& leaf = TestFixtureFactory<Fixture>::instance(name);
    root.add(leaf);
  }
};
 \end{c++}
\end{nodiff}

此时,\ascii{xUnit Mars}风格的测试套件\ascii{AutoStackSpec, AutoQueueSpec}与\ascii{Google Test}一同执行。

\begin{nodiff}{xUnit Mars执行结果}
 \begin{c++}
[==========] 6 test cases.
[----------] 3 tests from StackSpec
[ RUN      ] apply stack with pop 0 time
[       OK ] apply stack with pop 0 time(0 us)
[ RUN      ] apply stack with pop 1 time
[       OK ] apply stack with pop 1 time(0 us)
[ RUN      ] apply stack with pop 2 times
[       OK ] apply stack with pop 2 times(0 us)
[----------] 3 tests from StackSpec(0 us total)

[----------] 3 tests from QueueSpec
[ RUN      ] apply pop 0 time
[       OK ] apply pop 0 time(0 us)
[ RUN      ] apply pop 1 time
[       OK ] apply pop 1 time(0 us)
[ RUN      ] apply pop 2 times
[       OK ] apply pop 2 times(0 us)
[----------] 3 tests from QueueSpec(0 us total)

[==========] 6 test cases.
[  TOTAL   ] PASS: 6  FAILURE: 0  ERROR: 0  TIME: 3 us
 \end{c++}
\end{nodiff}

\end{content}

\section{中生代}

\begin{content}

\subsection{引入DSL}

\begin{nodiff}{test/mars/auto/AutoSpec.cc}
 \begin{c++}
#define GTEST_DONT_DEFINE_TEST 1
#include <gtest/gtest.h>
#include <mars/def/FixtureDef.h>
#include <mars/def/TestDef.h>
#include <stack>

namespace {
  FIXTURE(StackSpec) {
    std::stack<int> v;

    SETUP {
      v.push(1);
      v.push(2);
    }

    TEST("apply pop 0 time") {
      ASSERT_EQ(2, v.top());
    }

    TEST("apply pop 1 time") {
      v.pop();
      ASSERT_EQ(1, v.top());
    }

    TEST("apply pop 2 times") {
      v.pop();
      v.pop();
      ASSERT_TRUE(v.empty());
    }
  };
}
 \end{c++}
\end{nodiff}

\subsubsection{实现FixtureDef}

\begin{nodiff}{include/mars/dsl/FixtureDef.h}
 \begin{c++}
#include <mars/auto/AutoTestFixture.h>
#include <mars/core/TestFixture.h>
#include <mars/util/Symbol.h>
#include <mars/util/Self.h>

#define FIXTURE(type)                        \
struct type;                                 \
static AutoTestFixture<type>                 \
  UNIQUE_NAME(auto_suite_)(STRINGIZE(type)); \
struct type : TestFixture, Self<type>

#define SETUP    void setUp() override
#define TEARDOWN void tearDown() override

#define SUPER_SETUP(super)    super::setUp()
#define SUPER_TEARDOWN(super) super::tearDown()
 \end{c++}
\end{nodiff}

\subsubsection{实现TestDef}

\begin{nodiff}{include/mars/dsl/TestDef.h}
 \begin{c++}
#include <mars/auto/AutoTestMethod.h>
#include <mars/util/Symbol.h>

#define __TEST_NAME(id) JOIN(test_, id)
#define __AUTO_TEST(id) JOIN(auto_test_, id)

#define __DEF_TEST(id, name)               \
private:                                   \
AutoTestMethod __AUTO_TEST(id)             \
  {id, name, &self_type::__TEST_NAME(id)}; \
public:                                    \
void __TEST_NAME(id)()

#define TEST(name) __DEF_TEST(UNIQUE_ID, name)
 \end{c++}
\end{nodiff}

\subsubsection{类型萃取}

\begin{nodiff}{include/mars/util/Self.h}
 \begin{c++}
template <typename T>
struct Self {
protected:
  using self_type = T;
};
 \end{c++}
\end{nodiff}

\subsection{发布API}

\begin{nodiff}{include/mars/mars.h}
 \begin{c++}
#include <mars/def/FixtureDef.h>
#include <mars/def/TestDef.h>
#include <mars/startup/Startup.h>
 \end{c++}
\end{nodiff}

\end{content}