\begin{savequote}[45mm]
\ascii{Any fool can write code that a computer can understand. Good programmers write code that humans can understand.}
\qauthor{\ascii{- Martin Flower}}
\end{savequote}

\chapter{自动发现} 
\label{ch:auto-register}

\begin{content}

\end{content}

% 太古代、元古代、古生代、中生代和新生代.

\section{太古代}

\begin{content}

\subsection{xUnit Mars用例}

基于目前\ascii{xUnit Mars}的框架实现,已经具备条件设计\ascii{xUnit Mars}风格的测试用例了。目前,\ascii{xUnit Mars}缺失\ascii{DSL}的基本能力,此时用例命名遵循\cpp{}函数命名的规范。同时,继承\ascii{TestFixture},根据用例场景的实际需求,覆写相关的成员函数\ascii{setUp, tearDown}。

另外,\ascii{xUnit Mars}也缺失\emph{断言}的基本特性,目前暂时使用\ascii{Google Test}提供的基本特性完成。例如,遵循\ascii{xUnit Mars}风格,设计实现了\ascii{StackSpec}的测试用例套件。

\begin{nodiff}{test/mars/auto/StackSpec.cc}
 \begin{c++}
#include <gtest/gtest.h>
#include <mars/core/TestFixture.h>
#include <stack>

namespace {
  struct StackSpec : TestFixture {
    std::stack<int> v;

    void setUp() override {
      v.push(1);
      v.push(2);
    }

    void apply_pop_0_time() {
      ASSERT_EQ(2, v.top());
    }

    void apply_pop_1_time() {
      v.pop();
      ASSERT_EQ(1, v.top());
    }

    void apply_pop_2_times() {
      v.pop();
      v.pop();
      ASSERT_TRUE(v.empty());
    }
  };
}
 \end{c++}
\end{nodiff}

\subsection{手动注册}

目前,\ascii{xUnit Mars}缺失用例的\emph{自动注册},或\emph{自动发现}的基本功能。此时,为了能够运行\ascii{StackSpec}中的\emph{测试函数},目前只能手动注册。

\begin{nodiff}{test/mars/auto/StackSpec.cc}
 \begin{c++}
#include <gtest/gtest.h>
#include <mars/core/TestSuite.h>

// ...

TEST(ManualRegisterSpec, run_stack_spec) {
  TestSuite suite("StackSpec");
  suite.add(new TestMethod<StackSpec>(&StackSpec::apply_pop_0_time));
  suite.add(new TestMethod<StackSpec>(&StackSpec::apply_pop_1_time));
  suite.add(new TestMethod<StackSpec>(&StackSpec::apply_pop_2_times));
}
 \end{c++}
\end{nodiff}

\subsection{手动运行}

目前,\ascii{xUnit Mars}也缺失用例\emph{自动运行}的基本功能,需要手动完成用例的运行环境的搭建和启动过程。搭建运行环境,主要包括创建\ascii{TestResult},及其监听\ascii{TestResult}变化的一个或多个监听器\ascii{TestListener}。例如,此处仅提供一个监听器\ascii{TextProgress},用于监测\ascii{xUnit Mars}内部运行是否符合预期行为。

\begin{nodiff}{test/mars/auto/StackSpec.cc}
 \begin{c++}
#include <gtest/gtest.h>
#include <mars/core/TestResult.h>
#include <mars/listener/text/TextProgress.h>
#include <sstream>

// ...

namespace {
  struct ManualRegisterSpec : testing::Test {
  protected:
    ManualRegisterSpec() : progress(ss) {
      result.addListener(progress);
    }

    void run(::Test& test) {
      result.runRootTest(test);
    }

    void assertOutput(const char* output) {
      ASSERT_EQ(ss.str(), output);
    }

  protected:
    std::ostringstream ss;
    TestResult result;
    TextProgress progress;
  };
}

TEST_F(ManualRegisterSpec, run_stack_spec) {
  TestSuite suite("StackSpec");
  suite.add(new TestMethod<StackSpec>(&StackSpec::apply_pop_0_time));
  suite.add(new TestMethod<StackSpec>(&StackSpec::apply_pop_1_time));
  suite.add(new TestMethod<StackSpec>(&StackSpec::apply_pop_2_times));

  run(suite);
  assertOutput("starting...\n***\nend.\n");
}
 \end{c++}
\end{nodiff}

测试通过。

\end{content}

\section{元古代}

\begin{content}

\subsection{自动化}

手动注册\ascii{TestMethod}实在是太笨拙了,必须引入自动化的实现机制。重构既有的测试用例,并使其失败。首当其冲,将\ascii{ManualRegisterSpec}重命名为\ascii{AutoStackSpec},以示意图。

在\ascii{AutoStackSpec}中,创建用例树的根节点\ascii{root},并使用静态工厂方法\ascii{suite}完成初始化。在\ascii{suite}实现中,它试图从仓库\ascii{TestMethodRegistry<StackSpec>}的单例中获取所有测试方法。接下来,核心问题转变为:如何自动化地将\ascii{StackSpec}中的所有测试方法注册到\ascii{TestMethodRegistry<StackSpec>}单例之中。

\begin{nodiff}{test/mars/auto/StackSpec.cc}
 \begin{c++}
#include <gtest/gtest.h>
#include <mars/core/TestResult.h>
#include <mars/listener/text/TextProgress.h>
#include <mars/auto/TestMethodRegistry.h>
#include <sstream>

namespace {
  struct AutoStackSpec : testing::Test {
  protected:
    AutoStackSpec() : progress(ss), root(suite()) {
      result.addListener(progress);
    }

    ~AutoStackSpec() {
      delete root;
    }

    void assertOutput(const char* output) {
      ASSERT_EQ(ss.str(), output);
    }

  private:
    static ::Test* suite();

  protected:
    std::ostringstream ss;
    TestResult result;
    TextProgress progress;
    ::Test* root;
  };
  
  ::Test* AutoStackSpec::suite() {
    return TestMethodRegistry<StackSpec>::instance().suite();
  }
}

TEST_F(AutoStackSpec, run_stack_spec) {
  result.runRootTest(*root);
  assertOutput("starting...\n***\nend.\n");
}
 \end{c++}
\end{nodiff}

为了篇幅所限,后文仅展示\ascii{AutoStackSpec::suite}的重构效果,\ascii{AutoStackSpec}的整体逻辑不再详细罗列,请自行在此查阅上下文。

\subsection{空仓库}

应用伪实现,迁移\ascii{TestMethodRegistry}至产品实现时,发现无法搬迁\ascii{StackSpec}的相关成员函数至\ascii{TestMethodRegistry}之中,因为\ascii{StackSpec}属于测试领域,\ascii{TestMethodRegistry}属于产品领域。

因此,实现\ascii{TestMethodRegistry::suite}工厂方法时,暂且返回一个空的\ascii{TestSuite}实例,此时只能通过编译,但不能通过测试。

\begin{nodiff}{include/mars/auto/TestMethodRegistry.h}
 \begin{c++}
#include <mars/core/TestSuite.h>
#include <mars/util/Singleton.h>

GENERIC_SINGLETON(TestMethodRegistry, Fixture) {
  Test* suite() const {
    return new TestSuite();
  }
};
 \end{c++}
\end{nodiff}

\ascii{TestMethodRegistry}是一个单例,但与普通单例存在微妙的差异,它存在一个泛型参数\ascii{Fixture}。更为特殊地,在整个\ascii{TestMethodRegistry}类域中,都未曾使用过该泛型参数。

事实上,泛型参数\ascii{Fixture}的存在,将会自动地为每个\ascii{Fixture}创建一个单例。如此,对于每个\ascii{Fixture}类型,都会存在一个唯一的、独立的、隔离的存储空间。例如,在\ascii{StackSpec.cc}中,将自动创建一个\ascii{TestMethodRegistry<StackaSpec>}的单例。

\subsubsection{实用宏:单例}

实现单例遵循固有的实现模式,为了消除重复设计,提取\ascii{Singleton}的公共实现,并放在基础设施之中。引入\ascii{Singleton}将免费得到\ascii{instance}的公共方法实现。切忌在每个单例实现中,都复制拷贝一份单例实现;否则将引入严重的重复设计。

为了方便用户快速创建单例类,此处提供两个实用的宏定义。其中,\ascii{SINGLETON}用于创建普通的单例,而\ascii{GENERIC\_SINGLETON}创建带有泛型参数的单例。

\begin{nodiff}{include/mars/util/Singleton.h}
 \begin{c++}
template<typename T>
struct Singleton {
  static T& instance() {
    static T inst;
    return inst;
  }

protected:
  Singleton() {}
};

#define SINGLETON(type) \
  struct type : Singleton<type>

#define GENERIC_SINGLETON(type, ...)         \
  template<typename __VA_ARGS__>             \
  struct type : Singleton<type<__VA_ARGS__>>
 \end{c++}
\end{nodiff}

\subsection{代码注入}

为了能够通过测试,必须提供一种自动注册的实现机制,将\ascii{StackSpec}的所有测试方法自动地注册到\ascii{TestMethodRegistry<StackSpec>}单例之中。但是,由于\cpp{}缺失反射的机制,发现测试方法变得相当棘手。

应用\emph{代码注入}技术,当生成\ascii{StackSpec}实例时,将自动执行这些注入的代码,随将\ascii{StackSpec}之中的所有测试方法自动地注册到\ascii{TestMethodRegistry<StackSpec>}单例之中。

此时,需要改写\ascii{AutoStackSpec::suite}的实现,它创建了一个局部的\ascii{StackSpec}实例,此刻将触发完成所有\ascii{StackSpec}测试方法的自动注册。

众所周知,\ascii{StackSpec}实例会被经常创建和销毁。因此,\ascii{TestMethodRegistry<StackSpec>}仓库实现必须提供一种预先的、轻量级的准入机制,禁止重复,或覆盖行为发生,留待后续处理。

\begin{nodiff}{test/mars/auto/StackSpec.cc}
 \begin{c++}
#include <mars/core/TestFixture.h>
#include <mars/auto/AutoTestMethod.h>
#include <stack>

namespace {
  struct StackSpec : TestFixture {
    std::stack<int> v;

    void setUp() override {
      v.push(1);
      v.push(2);
    }

    void apply_pop_0_time() {
      ASSERT_EQ(2, v.top());
    }

    AutoTestMethod m1{1, "apply_pop_0_time", &StackSpec::apply_pop_0_time};

    void apply_pop_1_time() {
      v.pop();
      ASSERT_EQ(1, v.top());
    }

    AutoTestMethod m2{2, "apply_pop_1_time", &StackSpec::apply_pop_1_time};

    void apply_pop_2_times() {
      v.pop();
      v.pop();
      ASSERT_TRUE(v.empty());
    }

    AutoTestMethod m3{3, "apply_pop_0_time", &StackSpec::apply_pop_2_times};
  };

  ::Test* AutoStackSpec::suite() {
    StackSpec dummy; // register all test methods into registry.
    return TestMethodRegistry<StackSpec>::instance().suite();
  }
}
 \end{c++}
\end{nodiff}

\subsection{自动注册}

\ascii{AutoTestMethod}实现非常简单。在构造函数中,它生成一个\ascii{TestMethod}实例,并自动地注册到对应的\ascii{TestMethodRegistry<Fixture>}单例之中。

\begin{nodiff}{include/mars/auto/AutoTestMethod.h}
 \begin{c++}
#include <mars/auto/TestMethodRegistry.h>
#include <mars/core/TestMethod.h>

struct AutoTestMethod {
  template <typename Fixture>
  AutoTestMethod(int id, const char* name, void(Fixture::*method)()) {
    auto& registry = TestMethodRegistry<Fixture>::instance();
    registry.put(id, new TestMethod<Fixture>(method, name));
  }
};
 \end{c++}
\end{nodiff}

\subsection{仓库实现}

为了能够通过编译,首先在\ascii{TestMethodRegistry}中空实现注册接口\ascii{put}。待编译通过,接着为了通过测试,创建\ascii{std::map}的数据结构。一方面,基于抽象类型\ascii{Test},显式实现\ascii{put}方法,完成测试用例的注册。另一方面,工厂方法\ascii{suite}将当前已注册的所有\ascii{Test}实例打包为一个\ascii{TestSuite}实例,最终返回给客户。

在\ascii{TestMethodRegistry::put}实现中,预先判断\ascii{id}对应的用例是否已经存在。如果存在,则直接略过;否则,在\ascii{map}中创建相应的记录。

\begin{nodiff}{include/mars/auto/TestMethodRegistry.h}
 \begin{c++}
#include <mars/core/TestSuite.h>
#include <mars/util/Singleton.h>
#include <map>

GENERIC_SINGLETON(TestMethodRegistry, Fixture) {
  void put(int id, Test* test) {
    if (!exist(id)) {
      registry.emplace(id, test);
    }
  }

  Test* suite() const {
    auto suite = new TestSuite();
    for (auto& test : registry) {
      suite->add(test.second);
    }
    return suite;
  }

private:
  bool exist(int id) const {
    return registry.find(id) != registry.end();
  }

private:
  std::map<int, Test*> registry;
};
 \end{c++}
\end{nodiff}

此时,测试通过。

\end{content}

\section{古生代}

\begin{content}

\subsection{提炼工厂}

虽然能够自动注册\ascii{StackSpec}的测试方法,但是还是需要手动地调用\ascii{AutoStackSpec::suite}方法,自动注册\ascii{StackSpec}的测试方法,并创建\ascii{StackSpec}对应的\ascii{TestSuite}实例。

\begin{nodiff}{test/mars/auto/StackSpec.cc}
 \begin{c++}
namespace {
  ::Test* AutoStackSpec::suite() {
    StackSpec dummy; // register all test methods into registry.
    return TestMethodRegistry<StackSpec>::instance().suite();
  }
}
 \end{c++}
\end{nodiff}

当增加另外一组用例时,例如\ascii{QueueSpec}。

\begin{nodiff}{test/mars/auto/QueueSpec.cc}
 \begin{c++}
#include <mars/core/TestFixture.h>
#include <mars/auto/AutoTestMethod.h>
#include <queue>

namespace {
  struct QueueSpec : TestFixture {
    std::queue<int> q;

    void setUp() override {
      q.push(1);
      q.push(2);
    }

    void apply_pop_0_time() {
      ASSERT_EQ(1, q.front());
      ASSERT_EQ(2, q.back());
    }

    AutoTestMethod m1{1, "apply_pop_0_time", &QueueSpec::apply_pop_0_time};

    void apply_pop_1_time() {
      q.pop();
      ASSERT_EQ(2, q.front());
      ASSERT_EQ(2, q.back());
    }

    AutoTestMethod m2{2, "apply_pop_1_time", &QueueSpec::apply_pop_1_time};

    void apply_pop_2_times() {
      q.pop();
      q.pop();
      ASSERT_TRUE(q.empty());
    }

    AutoTestMethod m3{3, "apply_pop_2_times", &QueueSpec::apply_pop_2_times};
  };
}
 \end{c++}
\end{nodiff}

\ascii{AutoQueueSpec::suite}将重复创建逻辑,引入重复设计。

\begin{nodiff}{test/mars/auto/QueueSpec.cc}
 \begin{c++}
namespace {
  ::Test* AutoQueueSpec::suite() {
    QueueSpec dummy; // register all test methods into registry.
    return TestMethodRegistry<QueueSpec>::instance().suite();
  }
}
 \end{c++}
\end{nodiff}

\subsubsection{原子工厂}

基于上述观察,提取一个公共的泛型类\ascii{AutoTestFactory},消除重复设计。

\begin{nodiff}{test/mars/auto/QueueSpec.cc}
 \begin{c++}
#include <mars/auto/TestMethodRegistry.h>

struct Test;

template<typename Fixture>
struct AutoTestFactory {
  Test* make() {
    Fixture dummy; // register all test methods to registry.
    return TestMethodRegistry<Fixture>::instance().suite();
  }
};
 \end{c++}
\end{nodiff}

\subsubsection{复合工厂}

为了消除\ascii{AutoStackSpec, AutoQueueSpec}之间的重复逻辑,将\ascii{StackSpec}与\ascii{QueueSpec}统一打包,一并运行。

\begin{nodiff}{test/mars/auto/AutoRegisterSpec.cc}
 \begin{c++}
::Test* AutoRegisterSpec::suite() {
  AutoTestFactory<StackSpec> f1;
  AutoTestFactory<QueueSpec> f2;

  auto suite = new TestSuite;
  suite->add(f1.make());
  suite->add(f2.make());
  return suite;
}
 \end{c++}
\end{nodiff}

基于上述观察,提取关键抽象\ascii{TestFactory}。

\begin{nodiff}{include/mars/auto/TestFactory.h}
 \begin{c++}
struct Test;

struct TestFactory {
  virtual Test* make() = 0;
  virtual ~TestFactory() {}
};
 \end{c++}
\end{nodiff}

提取复合工厂\ascii{TestFactorySuite}。

\begin{nodiff}{include/mars/auto/TestFactorySuite.h}
 \begin{c++}
#include <mars/auto/TestFactory.h>
#include <vector>

struct TestFactorySuite : TestFactory {
  void addFactory(TestFactory&);

private:
  Test* make() override;

private:
  std::vector<TestFactory*> factories;
};
 \end{c++}
\end{nodiff}

\begin{nodiff}{src/mars/auto/TestFactorySuite.cc}
 \begin{c++}
#include <mars/auto/TestFactorySuite.h>
#include <mars/core/TestSuite.h>

void TestFactorySuite::addFactory(TestFactory& f) {
  factories.push_back(&f);
}

Test* TestFactorySuite::make() {
  auto suite = new TestSuite("All Tests");
  for (auto f : factories) {
    suite->add(f->make());
  }
  return suite;
}
 \end{c++}
\end{nodiff}

此时,重构原子工厂\ascii{AutoTestFactory},使其继承自\ascii{TestFactory}。

\begin{nodiff}{include/mars/auto/AutoTestFactory.h}
 \begin{c++}
#include <mars/auto/TestFactory.h> 
#include <mars/auto/TestMethodRegistry.h>

template<typename Fixture>
struct AutoTestFactory : TestFactory {
private:
  Test* make() override {
    Fixture dummy; // register all test methods to registry.
    return TestMethodRegistry<Fixture>::instance().suite();
  }
};
 \end{c++}
\end{nodiff}

\subsubsection{手动组建}

基于\ascii{TestFactory}的抽象,消除了\ascii{AutoTestFactory, TestFactorySuite}之间的一多之分。

\begin{nodiff}{test/mars/auto/AutoRegisterSpec.cc}
 \begin{c++}
::Test* AutoRegisterSpec::suite() {
  TestFactorySuite multi;
  AutoTestFactory<StackSpec> f1;
  AutoTestFactory<QueueSpec> f2;

  multi.addFactory(f1);
  multi.addFactory(f2);

  TestFactory& factory = multi;
  return factory.make();
}
 \end{c++}
\end{nodiff}

\subsection{自动组建}

但是,组建工厂树的逻辑相当讨厌,需要实现自动注册的机制,自动发现\ascii{StackSpec, QueeuSpec}对应的\ascii{AutoTestFactory},并逐一注册,减轻用户的负担。

\begin{nodiff}{test/mars/auto/AutoRegisterSpec.cc}
 \begin{c++}
::Test* AutoRegisterSpec::suite() {
  TestFactory& factory = TestFactorySuite::instance();
  return factory.make();
}
 \end{c++}
\end{nodiff}

\subsubsection{工厂仓库}

\begin{nodiff}{test/mars/auto/AutoRegisterSpec.cc}
 \begin{c++}
#include <mars/util/Singleton.h>
#include <mars/util/Keywords.h>
#include <mars/auto/TestFactory.h>
#include <vector>

SINGLETON(TestFactorySuite) EXTENDS(TestFactory) {
  void addFactory(TestFactory*);

private:
  Test* make() override;

private:
  std::vector<TestFactory*> factories;
};
 \end{c++}
\end{nodiff}

\subsection{代码注入}

\begin{nodiff}{test/mars/auto/StackSpec.cc}
 \begin{c++}
#include <mars/core/TestFixture.h>
#include <mars/auto/AutoTestSuite.h>
#include <stack>
#include <queue>

namespace {
  struct StackSpec;
  static AutoTestSuite<StackSpec> suite1;

  struct StackSpec : TestFixture {
    // ...
  };

  struct QueueSpec;
  static AutoTestSuite<QueueSpec> suite2;

  struct QueueSpec : TestFixture {
    // ...
  };  
}
 \end{c++}
\end{nodiff}

\subsection{自动注册}

\ascii{AutoTestSuite}在构造时,将原子工厂\ascii{AutoTestFactory}自动地注册到复合工厂\ascii{TestFactorySuite}的单例之中。

\begin{nodiff}{include/mars/auto/AutoTestSuite.h}
 \begin{c++}
#include <mars/auto/AutoTestFactory.h>
#include <mars/auto/TestFactorySuite.h>

template<typename Fixture>
struct AutoTestSuite {
  AutoTestSuite() {
    TestFactorySuite::instance().addFactory(atom);
  }

private:
  AutoTestFactory<Fixture> atom;
};
 \end{c++}
\end{nodiff}

测试通过。

\end{content}

\section{中生代}

\begin{content}

\subsection{代码注入}

\begin{nodiff}{test/mars/auto/StackSpec.cc}
 \begin{c++}
#include <mars/mars.h>
#include <stack>

namespace {
  FIXTURE(StackSpec) {
    std::stack<int> v;

    SETUP {
      v.push(1);
      v.push(2);
    }

    TEST("apply pop 0 time") {
      ASSERT_EQ(2, v.top());
    }

    TEST("apply pop 1 time") {
      v.pop();
      ASSERT_EQ(1, v.top());
    }

    TEST("apply pop 2 times") {
      v.pop();
      v.pop();
      ASSERT_TRUE(v.empty());
    }
  };
}
 \end{c++}
\end{nodiff}

\subsection{DSL: 领域描述语言}

\begin{nodiff}{include/mars/dsl/FixtureDef.h}
 \begin{c++}
#include <mars/auto/AutoTestSuite.h>
#include <mars/core/TestFixture.h>
#include <mars/util/Symbol.h>
#include <mars/util/Self.h>

#define FIXTURE(type)                                \
struct type;                                         \
static AutoTestSuite<type> UNIQUE_NAME(auto_suite_); \
struct type : TestFixture, Self<type>

#define SETUP    void setUp() override
#define TEARDOWN void tearDown() override

#define SUPER_SETUP(super)    super::setUp()
#define SUPER_TEARDOWN(super) super::tearDown()
 \end{c++}
\end{nodiff}

\begin{nodiff}{include/mars/dsl/TestDef.h}
 \begin{c++}
#include <mars/auto/AutoTestMethod.h>
#include <mars/util/Symbol.h>

#define __TEST_NAME(id) JOIN(test_, id)
#define __AUTO_TEST(id) JOIN(auto_test_, id)

#define __DEF_TEST(id, name)                 \
private:                                     \
  AutoTestMethod __AUTO_TEST(id)             \
    {id, name, &self_type::__TEST_NAME(id)}; \
public:                                      \
  void __TEST_NAME(id)()

#define TEST(name) __DEF_TEST(UNIQUE_ID, name)
 \end{c++}
\end{nodiff}

\begin{nodiff}{include/mars/dsl/TestDef.h}
 \begin{c++}
namespace {
  struct AutoRegisterSpec : testing::Test {
  protected:
    AutoRegisterSpec() : progress(ss), root(suite()) {
      result.addListener(progress);
    }

    ~AutoRegisterSpec() {
      delete root;
    }

    void assertOutput(const char* output) {
      ASSERT_EQ(ss.str(), output);
    }

  private:
    static ::Test* suite() {
      TestFactory& factory = TestFactorySuite::instance();
      return factory.make();
    }

  protected:
    std::ostringstream ss;
    TestResult result;
    TextProgress progress;
    ::Test* root;
  };
}

TEST_F(AutoRegisterSpec, run_all_xunit_mars_test) {
  result.runRootTest(*root);
  assertOutput("starting...\n******\nend.\n");
}
 \end{c++}
\end{nodiff}

\end{content}