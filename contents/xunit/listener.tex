\begin{savequote}[45mm]
\ascii{Any fool can write code that a computer can understand. Good programmers write code that humans can understand.}
\qauthor{\ascii{- Martin Flower}}
\end{savequote}

\chapter{监听器} 
\label{ch:listener}

\begin{content}

\end{content}

\section{收集器}

\begin{content}

\subsection{测试夹具}

\begin{nodiff}{test/mars/listener/TestCollectorSpec.cc}
 \begin{c++}
#include <mars/listener/TestCollector.h>
#include <mars/core/TestCase.h>
#include <mars/core/TestResult.h>
#include <gtest/gtest.h>

namespace {
  struct TestCollectorSpec : testing::Test {
  protected:
    void run(::Test& test) {
      test.run(result);
    }

  private:
    void SetUp() override {
      result.addListener(collector);
    }

  protected:
    TestResult result;
    TestCollector collector;
  };
}
 \end{c++}
\end{nodiff} 

\subsection{运行计数器}

\begin{nodiff}{test/mars/listener/TestCollectorSpec.cc}
 \begin{c++}
TEST_F(TestCollectorSpec, count_single_test_case) {
  TestCase test;
  run(test);
  ASSERT_EQ(1, collector.runCount());
}
 \end{c++}
\end{nodiff} 

\subsubsection{关键抽象}

提取监听器的抽象接口\ascii{TestListener},它监听测试用例生命周期的关键事件。

\begin{nodiff}{include/mars/core/TestListener.h}
 \begin{c++}
struct Test;

struct TestListener {
  virtual void startTestCase(const Test&) {}
  virtual ~TestListener() {}
};
  \end{c++}
\end{nodiff}

\subsubsection{实现收集器}

收集器\ascii{TestCollector.h}增加\emph{运行计数器}\ascii{numOfRuns},增加查询接口\ascii{runCount},覆写事件监听接口\ascii{startTestCase}。

\begin{nodiff}{include/mars/listener/TestCollector.h}
 \begin{c++}
#include <mars/core/TestListener.h>

struct TestCollector : TestListener {
  TestCollector();

  int runCount() const;

private:
  void startTestCase(const Test&) override;

private:
  int numOfRuns;
};
  \end{c++}
\end{nodiff}

在实现文件中,完成\emph{运行计数器}的初始化,查询接口\ascii{runCount}的代码逻辑,及其事件监听接口的实现逻辑。

\begin{nodiff}{include/mars/listener/TestCollector.h}
 \begin{c++}
#include <mars/listener/TestCollector.h>

TestCollector::TestCollector() : numOfRuns(0) {
}

int TestCollector::runCount() const {
  return numOfRuns;
}

void TestCollector::startTestCase(const Test&) {
  numOfRuns++;
}
 \end{c++}
\end{nodiff}

\subsubsection{注册监听器}

在\ascii{TestResult.h}中,增加私有数据\ascii{listeners},并对外公开监听器的注册接口\ascii{addListener}。此处,秉承“按照接口编程”的一致原则,使用抽象接口\ascii{TestListener},而非具体实现的\emph{TestCollector}。另外,为了简化内存管理的复杂度,\ascii{TestResult}不管理\ascii{TestListener}列表的内存,它仅仅持有它们的指针。

\begin{nodiff}{include/mars/listener/TestResult.h}
 \begin{c++}
#include <vector>

struct TestListener;

struct TestResult {
  void addListener(TestListener&);

  // ...

private:
  std::vector<TestListener*> listeners;
};
 \end{c++}
\end{nodiff}

在实现文件中,实现监听器注册接口\ascii{addListener}的实现逻辑。

\begin{nodiff}{src/mars/listener/TestResult.cc}
 \begin{c++}
#include <mars/core/TestResult.h>

// ...

void TestResult::addListener(TestListener& listener) {
  listeners.push_back(&listener);
}
 \end{c++}
\end{nodiff}

\subsubsection{广播事件}

在\ascii{TestResult.h}中,增加事件监听接口\ascii{startTestCase}。

\begin{nodiff}{include/mars/listener/TestResult.h}
 \begin{c++}
#include <vector>

struct Test;
struct TestListener;

struct TestResult {
  void addListener(TestListener&);
  void startTestCase(const Test&);

  // ...

private:
  std::vector<TestListener*> listeners;
};
 \end{c++}
\end{nodiff}

在实现文件中,实现事件监听接口\ascii{startTestCase}的代码逻辑。它广播给所有的监听器,执行相关逻辑。

\begin{nodiff}{src/mars/listener/TestResult.cc}
 \begin{c++}
#include <mars/core/TestResult.h>
#include <mars/core/TestListener.h>

// ...

void TestResult::addListener(TestListener& listener) {
  listeners.push_back(&listener);
}

void TestResult::startTestCase(const Test& test) {
  for (auto listener : listeners) {
    listener->startTestCase(test);
  }
}
 \end{c++}
\end{nodiff}

\subsubsection{通知事件}

在\ascii{TestCase::run}实现中,调用\ascii{startTestCase}通知\ascii{TestResult},测试用例已开始执行。

\begin{nodiff}{src/mars/core/TestCase.cc}
 \begin{c++}
#include <mars/core/TestCase.h>
#include <mars/core/TestResult.h>

// ...

#define PROTECT(method) \
  result.protect(Functor(this, &TestCase::method,  "in the "#method))

void TestCase::runBare(TestResult& result) {
  if (PROTECT(setUp)) {
    PROTECT(runTest);
  }
  PROTECT(tearDown);
}

void TestCase::run(TestResult& result) {
  result.startTestCase(*this);
  runBare(result);
}
 \end{c++}
\end{nodiff}

至此,测试通过。

\subsection{失败计数器}

构造一个失败的测试用例,收集器对外提供查询接口\ascii{failCount},用于查询执行失败的用例数目。

\begin{nodiff}{test/mars/listener/TestCollectorSpec.cc}
 \begin{c++}
namespace {
  struct FailureOnRunningTest : TestCase {
  private:
    void runTest() override {
      throw AssertionError("product.cc:57", "expected value == 2, but got 3");
    }
  };
}

TEST_F(TestCollectorSpec, throw_assertion_error_on_run_test) {
  FailureOnRunningTest test;
  run(test);
  ASSERT_EQ(1, collector.failCount());
}
 \end{c++}
\end{nodiff}

\subsubsection{关键抽象}

在抽象接口\ascii{TestListener}类型中增加事件监听接口\ascii{addFailure},负责监听测试用例执行失败的事件。

\begin{nodiff}{include/mars/core/TestListener.h}
 \begin{c++}
struct Test;
struct TestFailure;

struct TestListener {
  virtual void startTestCase(const Test&) {}
  virtual void addFailure(const TestFailure&) {}
  virtual ~TestListener() {}
};
  \end{c++}
\end{nodiff}

\subsubsection{实现收集器}

收集器\ascii{TestCollector.h}增加\emph{失败计数器}\ascii{numOfFails},增加失败数目的查询接口\ascii{failCount},及其覆写事件监听接口\ascii{addFailure}。

\begin{nodiff}{include/mars/listener/TestCollector.h}
 \begin{c++}
#include <mars/core/TestListener.h>

struct TestCollector : TestListener {
  TestCollector();

  int runCount() const;
  int failCount() const;

private:
  void startTestCase(const Test&) override;
  void addFailure(const TestFailure&) override;

private:
  int numOfRuns;
  int numOfFails;
};
 \end{c++}
\end{nodiff}

在实现文件中,完成\emph{失败计数器}的初始化,查询接口\ascii{failCount}的代码逻辑,事件监听接口\ascii{addFailure}的实现逻辑。

\begin{nodiff}{src/mars/listener/TestCollector.cc}
 \begin{c++}
#include <mars/listener/TestCollector.h>

TestCollector::TestCollector()
  : numOfRuns(0), numOfFails(0) {}

int TestCollector::runCount() const {
  return numOfRuns;
}

int TestCollector::failCount() const {
  return numOfFails;
}

void TestCollector::startTestCase(const Test&) {
  numOfRuns++;
}

void TestCollector::addFailure(const TestFailure&) {
  numOfFails++;
}
 \end{c++}
\end{nodiff}

\subsubsection{广播事件}

当执行\ascii{addFailure, addError}时,广播给所有的监听者执行相关逻辑。显而易见,此处存在明显的重复设计。

\begin{nodiff}{src/mars/listener/TestResult.cc}
 \begin{c++}
#include <mars/core/TestResult.h>
#include <mars/core/TestListener.h>

// ...

void TestResult::startTestCase(const Test& test) {
  for (auto listener : listeners) {
    listener->startTestCase(test);
  }
}

inline void TestResult::addFailure(std::string&& msg) {
  failures.emplace_back(std::move(msg), true);
  for (auto listener : listeners) {
    listener->addFailure(failures.back());
  }
}

inline void TestResult::addError(std::string&& msg) {
  failures.emplace_back(std::move(msg), false);
  for (auto listener : listeners) {
    listener->addFailure(failures.back());
  }
}

namespace {
  std::string msg(const char* why, const char* where, const char* what) {
    return std::string(why) + ' ' + where + '\n' + what;
  }
}

#define ON_FAIL(except)  addFailure(msg(except, f.where(), e.what()))
#define ON_ERROR(except) addError(msg(except, f.where(), e.what()))

// ...
 \end{c++}
\end{nodiff}

测试通过。

\subsubsection{消除重复}

为了消除重复,此处提取了一个宏函数\ascii{BOARDCAST}。另外,为了消除既有\ascii{addFailure, addError}的重复逻辑,将其合二为一。

\begin{nodiff}{src/mars/listener/TestResult.cc}
 \begin{c++}
#include <mars/core/TestResult.h>
#include <mars/core/TestListener.h>

// ...

#define BOARDCAST(action) for (auto listener : listeners) listener->action

void TestResult::startTestCase(const Test& test) {
  BOARDCAST(startTestCase(test));
}

void TestResult::addFailure(std::string& msg, bool failure) {
  failures.emplace_back(std::move(msg), failure);
  BOARDCAST(addFailure(failures.back()));
}

namespace {
  std::string msg(const char* why, const char* where, const char* what) {
    return std::string(why) + ' ' + where + '\n' + what;
  }
}

#define ON_FAIL(except)  addFailure(msg(except, f.where(), e.what()), true)
#define ON_ERROR(except) addFailure(msg(except, f.where(), e.what()), false)

// ...
 \end{c++}
\end{nodiff}

\subsection{错误计数器}

构造一个失败的测试用例,收集器对外提供查询接口\ascii{errorCount},用于查询执行错误的用例数目。

\begin{nodiff}{test/mars/listener/TestCollectorSpec.cc}
 \begin{c++}
namespace {
  struct ErrorOnRunningTest : TestCase {
   void runTest() override {
      throw std::exception();
    }
  };
}

TEST_F(TestCollectorSpec, throw_std_exception_on_run_test) {
  ErrorOnRunningTest test;
  run(test);
  ASSERT_EQ(1, collector.errorCount());
}
 \end{c++}
\end{nodiff}

\subsubsection{实现收集器}

收集器\ascii{TestCollector.h}增加\emph{错误计数器}\ascii{numOfErrors},及其查询接口\ascii{errorCount}。

\begin{nodiff}{include/mars/listener/TestCollector.h}
 \begin{c++}
#include <mars/core/TestListener.h>

struct TestCollector : TestListener {
  TestCollector();

  int runCount() const;
  int failCount() const;
  int errorCount() const;

private:
  void startTestCase(const Test&) override;
  void addFailure(const TestFailure&) override;

private:
  int numOfRuns;
  int numOfFails;
  int numOfErrors;
};
 \end{c++}
\end{nodiff}

在实现文件中,追加实现\ascii{numOfErrors}的初始化,实现查询接口\ascii{errorCount}的代码逻辑,重构\ascii{addFailure}的既有实现,根据\ascii{TestFailure}是否为错误,分别累加不同的计数器。。

\begin{nodiff}{include/mars/listener/TestCollector.h}
 \begin{c++}
#include <mars/listener/TestCollector.h>
#include <mars/except/TestFailure.h>

TestCollector::TestCollector()
  : numOfRuns(0)
  , numOfFails(0)
  , numOfErrors(0) {}

int TestCollector::runCount() const {
  return numOfRuns;
}

int TestCollector::failCount() const {
  return numOfFails;
}

int TestCollector::errorCount() const {
  return numOfErrors;
}

void TestCollector::startTestCase(const Test&) {
  numOfRuns++;
}

void TestCollector::addFailure(const TestFailure& f) {
  f.isFailure() ? numOfFails++ : numOfErrors++;
}
 \end{c++}
\end{nodiff}

\subsection{通过计数器}

构造一个失败的测试用例,收集器对外提供查询接口\ascii{passCount},用于查询执行通过的用例数目。

\begin{nodiff}{test/mars/listener/TestCollectorSpec.cc}
 \begin{c++}
#include <mars/core/TestSuite.h>

namespace {
  struct ErrorOnRunningTest : TestCase {
   void runTest() override {
      throw std::exception();
    }
  };
}

TEST_F(TestCollectorSpec, count_test_cases_from_collector) {
  TestSuite suite;
  suite.add(new TestCase);
  suite.add(new FailureOnRunningTest);

  run(suite);

  ASSERT_EQ(2, collector.runCount());
  ASSERT_EQ(1, collector.passCount());
  ASSERT_EQ(1, collector.failCount());
}
 \end{c++}
\end{nodiff}

\subsubsection{实现收集器}

收集器\ascii{TestCollector.h}增加\emph{通过计数器}\ascii{numOfPassed},及其标记位\ascii{lastFailed},增加查询接口\ascii{passCount}。其中,\ascii{lastFailed}用于标记最近的一个测试用例是否执行成功。

\begin{nodiff}{include/mars/listener/TestCollector.h}
 \begin{c++}
#include <mars/core/TestListener.h>

struct TestCollector : TestListener {
  TestCollector();

  int runCount() const;
  int failCount() const;
  int errorCount() const;
  int passCount() const;

private:
  void startTestCase(const Test&) override;
  void addFailure(const TestFailure&) override;

private:
  int numOfRuns;
  int numOfFails;
  int numOfErrors;
  int numOfPassed;
  bool lastFailed;
};
 \end{c++}
\end{nodiff}

在实现文件中,当收到\ascii{startTestCase}事件时,将\ascii{lastFailed}重新初始化为\ascii{false}。当收到\ascii{startTestCase}事件时,将\ascii{lastFailed}标记置为\ascii{true}。当收到\ascii{endTestCase}事件时,如果\ascii{lastFailed}为\ascii{false},则累计\emph{通过计数器}\ascii{numOfPassed}。

\begin{nodiff}{src/mars/listener/TestCollector.cc}
 \begin{c++}
#include <mars/listener/TestCollector.h>
#include <mars/except/TestFailure.h>

TestCollector::TestCollector()
  : numOfRuns(0)
  , numOfFails(0)
  , numOfErrors(0)
  , numOfPassed(0)
  , lastFailed(false) {}

int TestCollector::runCount() const {
  return numOfRuns;
}

int TestCollector::failCount() const {
  return numOfFails;
}

int TestCollector::errorCount() const {
  return numOfErrors;
}

int TestCollector::passCount() const {
  return numOfPassed;
}

void TestCollector::startTestCase(const Test&) {
  numOfRuns++;
  lastFailed = false;
}

void TestCollector::endTestCase(const Test&) {
  if (!lastFailed) numOfPassed++;
}

void TestCollector::addFailure(const TestFailure& f) {
  f.isFailure() ? numOfFails++ : numOfErrors++;
  lastFailed = true;
}
 \end{c++}
\end{nodiff}

\subsubsection{广播事件}

在\ascii{TestResult}中,增加\ascii{endTestCase}事件监听接口,它广播所有监听器该事件的到达消息。

\begin{nodiff}{src/mars/listener/TestResult.cc}
 \begin{c++}
#include <mars/core/TestResult.h>
#include <mars/core/TestListener.h>

// ...

void TestResult::addListener(TestListener& listener) {
  listeners.push_back(&listener);
}

#define BOARDCAST(action) for (auto listener : listeners) listener->action

void TestResult::startTestCase(const Test& test) {
  BOARDCAST(startTestCase(test));
}

void TestResult::endTestCase(const Test& test) {
  BOARDCAST(endTestCase(test));
}

void TestResult::addFailure(std::string&& msg, bool failure) {
  failures.emplace_back(std::move(msg), failure);
  BOARDCAST(addFailure(failures.back()));
}

namespace {
  std::string msg(const char* why, const char* where, const char* what) {
    return std::string(why) + ' ' + where + '\n' + what;
  }
}

#define ON_FAIL(except)  addFailure(msg(except, f.where(), e.what()), true)
#define ON_ERROR(except) addFailure(msg(except, f.where(), e.what()), false)

// ...
 \end{c++}
\end{nodiff}

\subsubsection{通知事件}

在\ascii{TestCase::run}实现中,追加实现\ascii{TestResult::endTestCase}事件的通知消息。

\begin{nodiff}{src/mars/core/TestCase.cc}
 \begin{c++}
#include <mars/core/TestCase.h>
#include <mars/core/TestResult.h>

// ...

#define PROTECT(method) \
  result.protect(Functor(this, &TestCase::method,  "in the "#method))

void TestCase::runBare(TestResult& result) {
  if (PROTECT(setUp)) {
    PROTECT(runTest);
  }
  PROTECT(tearDown);
}

void TestCase::run(TestResult& result) {
  result.startTestCase(*this);
  runBare(result);
  result.endTestCase(*this);  
}
 \end{c++}
\end{nodiff}

测试通过。

\end{content}

\section{状态查询器}

\begin{content}

存在一种场景,用户只关心用例是否执行成功。按照既有的实现逻辑,可以在\ascii{TestCollector}增加\ascii{successful}的状态查询接口。但是,考虑到此需求可能被用户单独使用,提炼该实现至一个新的监听器\ascii{TestStatus},其职责非常单一。

\subsection{测试夹具}

\begin{nodiff}{test/mars/listener/TestStatusSpec.cc}
 \begin{c++}
#include <mars/listener/TestStatus.h>
#include <mars/core/TestResult.h>
#include <mars/core/TestCase.h>
#include <gtest/gtest.h>

namespace {
  struct TestStatusSpec : testing::Test {
  protected:
    void run(::Test& test) {
      test.run(result);
    }

  private:
    void SetUp() override {
      result.addListener(status);
    }

  protected:
    TestResult result;
    TestStatus status;
  };
}
 \end{c++}
\end{nodiff}

\subsection{成功场景}

\begin{nodiff}{test/mars/listener/TestStatusSpec.cc}
 \begin{c++}
TEST_F(TestStatusSpec, should_be_successful) {
  TestCase test;
  run(test);
  ASSERT_TRUE(status.successful());
}
 \end{c++}
\end{nodiff}

\subsubsection{实现状态查询器}

\ascii{TestStatus}只用监听\ascii{addFailure}事件,对外提供状态查询接口\ascii{successful}。

\begin{nodiff}{include/mars/listener/TestStatus.h}
 \begin{c++}
struct TestStatus {
  TestStatus();
  bool successful() const;

private:
  bool succ;
};
 \end{c++}
\end{nodiff}

在实现文件中,初始状态下,\ascii{TestStatus}将其标记位\ascii{succ}置为\ascii{true},其状态查询接口\ascii{successful}直接返回标记位\ascii{succ}的当前值。

\begin{nodiff}{src/mars/listener/TestStatus.cc}
 \begin{c++}
#include <mars/listener/TestStatus.h>

TestStatus::TestStatus() : succ(true) {
}

bool TestStatus::successful() const {
  return succ;
}
 \end{c++}
\end{nodiff}

测试通过。

\subsection{失败场景}

\begin{nodiff}{test/mars/listener/TestStatusSpec.cc}
 \begin{c++}
namespace {
  struct FailureOnRunningTest : TestCase {
  private:
    void runTest() override {
      throw AssertionError("product.cc:57", "expected value == 2, but got 3");
    }
  };
}

TEST_F(TestStatusSpec, should_be_failed) {
  FailureOnRunningTest test;
  run(test);
  ASSERT_FALSE(status.successful());
}
 \end{c++}
\end{nodiff}

\subsubsection{实现状态查询器}

\ascii{TestStatus}增加事件监听接口\ascii{addFailure}。

\begin{nodiff}{include/mars/listener/TestStatus.h}
 \begin{c++}
#include <mars/core/TestListener.h>

struct TestStatus : TestListener {
  TestStatus();
  bool successful() const;

private:
  void addFailure(const TestFailure&) override;

private:
  bool succ;
};
 \end{c++}
\end{nodiff}

在实现文件中,一旦收到\ascii{addFailure}事件的通知消息,将标记位\ascii{succ}置为\ascii{false}。

\begin{nodiff}{src/mars/listener/TestStatus.cc}
 \begin{c++}
#include <mars/listener/TestStatus.h>

TestStatus::TestStatus() : succ(true) {
}

bool TestStatus::successful() const {
  return succ;
}

void TestStatus::addFailure(const TestFailure&) {
  succ = false;
}
 \end{c++}
\end{nodiff}

测试通过。

\end{content}

\section{错误列表}

\begin{content}

在遗留的系统实现中,\ascii{TestResult}通过暴露\ascii{listFailures}的查询接口,返回整个错误列表给用户。用户通过操作\ascii{std::vector}的数据结构,自行获取其感兴趣的信息。这样的设计,不仅导致客户端代码产生大量重复的\ascii{for}语句,而且使得\ascii{TestResult}的私有存储结构暴露到众多客户之下,丢失了基本的封装特性。

\subsection{测试夹具}

\begin{nodiff}{test/mars/listener/TestStatusSpec.cc}
 \begin{c++}
#include <mars/listener/FailureList.h>
#include <mars/core/TestCase.h>
#include <mars/core/TestResult.h>
#include <gtest/gtest.h>

struct FailureListSpec : testing::Test {
protected:
  void run(::Test& test) {
    test.run(result);
  }

private:
  void SetUp() override {
    result.addListener(list);
  }

protected:
  FailureLister list;
  TestResult result;
};
 \end{c++}
\end{nodiff}

\subsection{异常信息}

\begin{nodiff}{test/mars/listener/FailureListSpec.cc}
 \begin{c++}
namespace {
  struct FailureOnRunningTest : TestCase {
    const char* expectMsg() const {
      return "assertion fail in the runTest\n"
             "product.cc:57\n"
             "expected value == 2, but got 3";
    }

  private:
    void runTest() override {
      throw AssertionError("product.cc:57", "expected value == 2, but got 3");
    }
  };
}

TEST_F(FailureListerSpec, extract_except_msg_on_running_test_failed) {
  FailureOnRunningTest test;
  run(test);

  list.foreach([&test](auto& fail) {
    ASSERT_TRUE(fail.isFailure());
    ASSERT_EQ(test.expectMsg(), fail.getExceptionMsg());
  });
}
 \end{c++}
\end{nodiff}

\subsubsection{实现错误列表}

\ascii{FailureList}只用监听\ascii{addFailure}事件,对外提供查询接口\ascii{foreach}。

\begin{nodiff}{include/mars/listener/FailureList.h}
 \begin{c++}
#include <vector>
#include <mars/core/TestListener.h>
#include <mars/except/TestFailure.h>

struct TestFailure;

struct FailureList : TestListener {
  template <typename F>
  void foreach(F f) const {
    for (auto failure : failures) {
      f(*failure);
    }
  }

private:
  void addFailure(const TestFailure&) override;

private:
  std::vector<const TestFailure*> failures;
};
 \end{c++}
\end{nodiff}

在其实现文件中,实现\ascii{addFailure}的实现逻辑。

\begin{nodiff}{src/mars/listener/FailureList.cc}
 \begin{c++}
#include <mars/listener/FailureList.h>

void FailureList::addFailure(const TestFailure& f) {
  failures.push_back(&f);
}
 \end{c++}
\end{nodiff}

测试通过。此时,便可以安全删除\ascii{TestResult::getFailures}的查询接口,及其相关逻辑,职责搬迁完毕。

\end{content}

\section{文本进度条}

\begin{content}

\subsection{测试夹具}

\begin{nodiff}{test/mars/listener/text/TextProgressSpec.cc}
 \begin{c++}
#include <mars/listener/text/TextProgress.h>
#include <mars/core/TestResult.h>
#include <mars/core/TestCase.h>
#include <mars/core/TestSuite.h>
#include <mars/except/AssertionError.h>
#include <gtest/gtest.h>
#include <sstream>

namespace {
  struct TextProgressSpec : testing::Test {
  protected:
    TextProgressSpec() : progress(ss) {
      result.addListener(progress);
    }

    void run(::Test& test) {
      result.startTestRun(test);
      test.run(result);
      result.endTestRun(test);
    }

    void assertOutput(const char* output) {
      ASSERT_EQ(ss.str(), output);
    }

  protected:
    TestResult result;
    std::ostringstream ss;
    TextProgress progress;
  };
}
 \end{c++}
\end{nodiff}

\subsection{成功场景}

\begin{nodiff}{test/mars/listener/text/TextProgressSpec.cc}
 \begin{c++}
TEST_F(TextProgressSpec, single_test_case_is_successful) {
  TestCase test;
  run(test);
  assertOutput("starting...\n*\nend.\n");
}
 \end{c++}
\end{nodiff}

一个测试用例,使用一个\ascii{*}表示,打印效果如下:

\begin{nodiff}{输出效果}
 \begin{c++}
starting...
*
end.
 \end{c++}
\end{nodiff}

\subsubsection{监听事件}

\begin{nodiff}{include/mars/core/TestListener.h}
 \begin{c++}
struct Test;
struct TestFailure;

struct TestListener {
  virtual void startTestRun(const Test&) {}
  virtual void endTestRun(const Test&) {}

  virtual void startTestCase(const Test&) {}
  virtual void endTestCase(const Test&) {}

  virtual void addFailure(const TestFailure&) {}

  virtual ~TestListener() {}
};
 \end{c++}
\end{nodiff}

\subsubsection{入口函数}

\begin{nodiff}{include/mars/core/TestResult.h}
 \begin{c++}
#include <mars/core/TestResult.h>
#include <mars/core/TestListener.h>

#define BOARDCAST(action) \
  for (auto listener : listeners) listener->action

void TestResult::startTestRun(const Test& test) {
  BOARDCAST(startTestRun(test));
}

void TestResult::endTestRun(const Test& test) {
  BOARDCAST(endTestRun(test));
}
 \end{c++}
\end{nodiff}

\subsubsection{实现文本进度条}

\begin{nodiff}{include/mars/listener/text/TextProgress.h}
 \begin{c++}
#include <mars/core/TestListener.h>
#include <ostream>

struct TextProgress : TestListener {
  explicit TextProgress(std::ostream&);

private:
  void startTestRun(const Test&) override;
  void startTestCase(const Test&) override;
  void endTestRun(const Test&) override;

private:
  std::ostream& out;
};
 \end{c++}
\end{nodiff}

\begin{nodiff}{src/mars/listener/text/TextProgress.cc}
 \begin{c++}
#include <mars/listener/text/TextProgress.h>

TextProgress::TextProgress(std::ostream& out)
  : out(out) {}

void TextProgress::startTestRun(const Test&) {
  out << "starting..."
      << std::endl;
}


void TextProgress::startTestCase(const Test&) {
  out << '*';
}

void TextProgress::endTestRun(const Test&) {
  out << std::endl
      << "end."
      << std::endl
      << std::flush;
}
 \end{c++}
\end{nodiff}

\subsection{失败场景}

\begin{nodiff}{test/mars/listener/text/TextProgressSpec.cc}
 \begin{c++}
namespace {
  struct FailureOnRunningTest : TestCase {
  private:
    void runTest() override {
      throw AssertionError("product.cc:57", "expected value == 2, but got 3");
    }
  };
}

TEST_F(TextProgressSpec, contains_failed_test_case) {
  TestSuite suite;
  suite.add(new TestCase);
  suite.add(new FailureOnRunningTest);

  run(suite);
  assertOutput("starting...\n**F\nend.\n");
}
 \end{c++}
\end{nodiff}

一个测试用例,使用一个\ascii{*}表示,打印效果如下:

\begin{nodiff}{输出效果}
 \begin{c++}
starting...
**F
end.
 \end{c++}
\end{nodiff}

\subsubsection{覆写addFailure}

\begin{nodiff}{include/mars/listener/text/TextProgress.h}
 \begin{c++}
#include <mars/core/TestListener.h>
#include <ostream>

struct TextProgress : TestListener {
  explicit TextProgress(std::ostream&);

private:
  void startTestRun(const Test&) override;
  void startTestCase(const Test&) override;
  void addFailure(const TestFailure&) override;  
  void endTestRun(const Test&) override;

private:
  std::ostream& out;
};
 \end{c++}
\end{nodiff}

\begin{nodiff}{src/mars/listener/text/TextProgress.cc}
 \begin{c++}
#include <mars/listener/text/TextProgress.h>

TextProgress::TextProgress(std::ostream& out)
  : out(out) {}

void TextProgress::startTestRun(const Test&) {
  out << "starting..."
      << std::endl;
}

void TextProgress::startTestCase(const Test&) {
  out << '*';
}

void TextProgress::addFailure(const TestFailure&) {
  out << 'F';
}

void TextProgress::endTestRun(const Test&) {
  out << std::endl
      << "end."
      << std::endl
      << std::flush;
}
 \end{c++}
\end{nodiff}

\subsection{错误场景}

\begin{nodiff}{test/mars/listener/text/TextProgressSpec.cc}
 \begin{c++}
namespace {
  struct ErrorOnRunningTest : TestCase {
   void runTest() override {
      throw std::exception();
    }
  };
}

TEST_F(TextProgressSpec, contains_error_test_case) {
  TestSuite suite;
  suite.add(new TestCase);
  suite.add(new ErrorOnRunningTest);

  run(suite);
  assertOutput("starting...\n**E\nend.\n");
}
 \end{c++}
\end{nodiff}

一个测试用例,使用一个\ascii{*}表示,打印效果如下:

\begin{nodiff}{输出效果}
 \begin{c++}
starting...
**E
end.
 \end{c++}
\end{nodiff}

\subsubsection{覆写addFailure}

\begin{nodiff}{src/mars/listener/text/TextProgress.cc}
 \begin{c++}
#include <mars/listener/text/TextProgress.h>

// ...

void TextProgress::addFailure(const TestFailure& f) {
  out << (f.isFailure() ? 'F' : 'E');
}
 \end{c++}
\end{nodiff}

测试通过。

\end{content}

\section{时间统计器}

\begin{content}

\subsection{时间测试}

\begin{nodiff}{test/mars/listener/collector/TimeCollectorSpec.cc}
 \begin{c++}
#include <mars/listener/collector/TimeCollector.h>
#include <mars/core/TestResult.h>
#include <mars/core/TestCase.h>
#include <mars/core/TestSuite.h>
#include <gtest/gtest.h>

namespace {
  struct TimeCollectorSpec : testing::Test {
  protected:
    TimeCollectorSpec() {
      result.addListener(clock);
    }

    void run(::Test& test) {
      result.startTestRun(test);
      test.run(result);
      result.endTestRun(test);
    }

    void assertTime(const TimeVal& val) {
      timeval max{0, 100};  // 100 us
      ASSERT_LT(val, TimeVal::by(max));
    }

  protected:
    TestResult result;
    TimeCollector clock;
  };
}

TEST_F(TimeCollectorSpec, should_be_successful) {
  TestSuite suite;
  suite.add(new TestCase);

  run(suite);

  assertTime(clock.caseTime());
  assertTime(clock.suiteTime());
  assertTime(clock.totalTime());
}
 \end{c++}
\end{nodiff}

\subsection{实现统计器}

\begin{nodiff}{include/mars/listener/collector/TimeCollector.h}
 \begin{c++}
#include <mars/core/TestListener.h>
#include <mars/util/TimeVal.h>
#include <stack>

struct TimeCollector : TestListener {
  ~TimeCollector();

  TimeVal caseTime() const;
  TimeVal suiteTime() const;
  TimeVal totalTime() const;

private:
  void startTestRun(const Test&) override;
  void endTestRun(const Test&) override;

  void startTestSuite(const Test&) override;
  void endTestSuite(const Test&) override;

  void startTestCase(const Test&) override;
  void endTestCase(const Test&) override;

private:
  void start();
  void end(TimeVal&);
  void pop();

private:
  TimeVal casetime;
  TimeVal suitetime;
  TimeVal totaltime;

  struct Timer;
  std::stack<Timer*> stack;
};
 \end{c++}
\end{nodiff}

\subsubsection{计时器}

\begin{nodiff}{src/mars/listener/collector/TimeCollector.cc}
 \begin{c++}
struct TimeCollector::Timer {
  Timer() {
    start.now();
  }

  void elapsed(TimeVal& last) const {
    last.now();
    last -= start;
  }

private:
  TimeVal start;
};
 \end{c++}
\end{nodiff}

\subsubsection{时间值}

\begin{nodiff}{include/mars/util/TimeVal.h}
 \begin{c++}
#include <time.h>
#include <mars/util/StructWrapper.h>

STRUCT_WRAPPER(TimeVal, timeval) {
  TimeVal();
  void now();

  bool operator<(const TimeVal&) const;
  TimeVal& operator-=(const TimeVal&);
};
 \end{c++}
\end{nodiff}

\begin{nodiff}{src/mars/util/TimeVal.cc}
 \begin{c++}
#include <mars/util/TimeVal.h>
#include <sys/time.h>

TimeVal::TimeVal() {
  tv_sec  = 0;
  tv_usec = 0;
}

void TimeVal::now() {
  gettimeofday(this, 0);
}

bool TimeVal::operator<(const TimeVal& rhs) const {
  if (tv_sec < rhs.tv_sec) return true;
  if (rhs.tv_sec < tv_sec) return false;
  return tv_usec < rhs.tv_usec;
}

TimeVal& TimeVal::operator-=(const TimeVal& start) {
  while (tv_usec < start.tv_usec) {
    tv_usec += 1000000;
    tv_sec--;
  }
  tv_usec -= start.tv_usec;
  tv_sec  -= start.tv_sec;
  return *this;
}
 \end{c++}
\end{nodiff}

\subsubsection{实用宏}

\begin{nodiff}{include/mars/util/StructWrapper.h}
 \begin{c++}
template<typename From, typename To>
struct StructWrapper : protected From {
  static const To& by(const From& from) {
    static_assert(sizeof(From) == sizeof(To), "");
    return (const To&) from;
  }

  static To& by(From& from) {
    static_assert(sizeof(From) == sizeof(To), "");
    return (To&) from;
  }
};

#define STRUCT_WRAPPER(To, From) \
  struct To : StructWrapper<From, To>
 \end{c++}
\end{nodiff}

\subsubsection{统计时间}

\begin{nodiff}{src/mars/listener/collector/TimeCollector.cc}
 \begin{c++}
inline void TimeCollector::start() {
  stack.push(new Timer);
}

inline void TimeCollector::pop() {
  auto top = stack.top();
  stack.pop();
  delete top;
}

inline void TimeCollector::end(TimeVal& val) {
  stack.top()->elapsed(val);
  pop();
}

TimeCollector::~TimeCollector() {
  while (!stack.empty()) {
    pop();
  }
}

void TimeCollector::startTestRun(const Test&) {
  start();
}

void TimeCollector::endTestRun(const Test&) {
  end(totaltime);
}

void TimeCollector::startTestSuite(const Test&) {
  start();
}

void TimeCollector::endTestSuite(const Test&) {
  end(suitetime);
}

void TimeCollector::startTestCase(const Test&) {
  start();
}

void TimeCollector::endTestCase(const Test&) {
  end(totaltime);
}
 \end{c++}
\end{nodiff}

\subsubsection{查询接口}

\begin{nodiff}{src/mars/listener/collector/TimeCollector.cc}
 \begin{c++}
TimeVal TimeCollector::caseTime() const {
  return casetime;
}

TimeVal TimeCollector::suiteTime() const {
  return suitetime;
}

TimeVal TimeCollector::totalTime() const {
  return totaltime;
}
 \end{c++}
\end{nodiff}

\subsubsection{广播事件}

\begin{nodiff}{src/mars/core/TestResult.cc}
 \begin{c++}
#include <mars/core/TestResult.h>
#include <mars/core/TestListener.h>

#define BOARDCAST(action) \
  for (auto listener : listeners) listener->action

void TestResult::startTestSuite(const Test& test) {
  BOARDCAST(startTestSuite(test));
}

void TestResult::endTestSuite(const Test& test) {
  BOARDCAST(endTestSuite(test));
}

// ...
 \end{c++}
\end{nodiff}

\subsubsection{通知事件}

\begin{nodiff}{src/mars/core/TestSuite.cc}
 \begin{c++}
#include <mars/core/TestSuite.h>
#include <mars/core/TestResult.h>

template <typename F>
inline void TestSuite::foreach(F f) const {
  for (auto test : tests) {
    f(test);
  }
}

void TestSuite::runBare(TestResult& result) {
  foreach([&result](auto test) {
    test->run(result);
  });
}

void TestSuite::run(TestResult& result) {
  result.startTestSuite(*this);
  runBare(result);
  result.endTestSuite(*this);
}

// ...
 \end{c++}
\end{nodiff}

\end{content}

\section{生命周期}

\begin{content}

\begin{nodiff}{include/mars/core/TestListener.h}
 \begin{c++}
struct Test;
struct TestFailure;

struct TestListener {
  virtual void startTestRun(const Test&) {}
  virtual void endTestRun(const Test&) {}

  virtual void startTestSuite(const Test&) {}
  virtual void endTestSuite(const Test&) {}

  virtual void startTestCase(const Test&) {}
  virtual void endTestCase(const Test&) {}

  virtual void addFailure(const TestFailure&) {}

  virtual ~TestListener() {}
};
 \end{c++}
\end{nodiff}

\end{content}

