\begin{savequote}[45mm]
\ascii{Any fool can write code that a computer can understand. Good programmers write code that humans can understand.}
\qauthor{\ascii{- Martin Flower}}
\end{savequote}

\chapter{解耦合} 
\label{ch:decoupling}

\begin{content}

\end{content}

\section{关键抽象}

\begin{content}

分析\ascii{TestCase}实现,\ascii{TestCase::setUp, TestCase::runTest, TestCase::tearDown}宿主于\ascii{TestCase},而异常处理逻辑主要宿主于\ascii{TestResult}。

根据迪米特法则,这段异常处理的逻辑应搬迁至\ascii{TestResult}。当捕获异常后,直接更新到自己的私有数据域,使得\ascii{TestResult::onRun, TestResult::onFail, TestResult::onError}等监听接口完全私有化。

但是,\ascii{TestResult}如何调用到\ascii{TestCase::setUp, TestCase::runTest, TestCase::tearDown}成员函数呢?此时似乎陷入窘境,毕竟\ascii{TestCase}与\ascii{TestResult}之间的耦合太过紧密,拆分显得不是那么容易。

解除\ascii{TestCase}与\ascii{TestResult}之间的耦合迫在眉睫。

\subsection{引入TestCaseFunctor}

为了彻底解开\ascii{TestCase::protect}对\ascii{TestCase}的依赖,引入关键抽象\ascii{TestCaseFunctor},相对\ascii{TestCase::Method}的简单抽象,它更加稳定,更加具有可扩展性(可以携带更多元数据,例如增加\ascii{getTestName}接口)。

\begin{leftbar}
 \begin{c++}[caption={\ttfamily{include/mars/core/internal/TestCaseFunctor.h}}]
#include <string>

struct TestCaseFunctor {
  virtual ~TestCaseFunctor() {}
  virtual bool operator()() const = 0;
};
 \end{c++}
\end{leftbar}

\subsection{实现TestCaseFunctor}

在\ascii{TestCase.h}的文件中,修改\ascii{TestCase::protect}的声明,让其依赖于更稳定的抽象\ascii{TestCaseFunctor},而非简单抽象\ascii{TestCase::Method},因为\ascii{TestCaseFunctor}不包含\ascii{TestCase}的任何信息。

\begin{leftbar}
 \begin{c++}[caption={\ttfamily{include/mars/core/TestCase.h}}]
#include "mars/core/Test.h"

struct TestCaseFunctor;

struct TestCase : Test {
  using Test::Test;

private:
  int countTestCases() const override;
  void run(TestResult&) override;

private:
  virtual void setUp() {}
  virtual void runTest() {}
  virtual void tearDown() {}

private:
  bool protect(TestResult&, const TestCaseFunctor&);
};
 \end{c++}
\end{leftbar}

在\ascii{TestCase.cc}实现文件中,在匿名命名空间内定义实现\ascii{Functor},它是\ascii{TestCaseFunctor}背后运行的真正实现,它包装实现了原来\ascii{TestCase::Method}的间接调用。

\begin{leftbar}
 \begin{c++}[caption={\ttfamily{src/mars/core/TestCase.cc}}]
#include "mars/core/TestCase.h"
#include "mars/core/TestResult.h"
#include "mars/core/internal/TestCaseFunctor.h"
#include "mars/except/AssertionError.h"

namespace {
  struct Functor : TestCaseFunctor {
    using Method = void(TestCase::*)();

    Functor(TestCase* self, Method method)
      : self(self), method(method) {
    }

  private:
    bool operator()() const override {
      (self->*method)();
      return true;
    }

  private:
    TestCase* self;
    Method method;
  };
}

bool TestCase::protect(TestResult& result, const TestCaseFunctor& f) {
  try {
    return f();
  } catch (const AssertionError&) {
    result.onFail();
  } catch (const std::exception&) {
    result.onError();
  } catch (...) {
    result.onError();
  }
  return false;
}

void TestCase::run(TestResult& result) {
  if (protect(result, Functor(this, &TestCase::setUp))) {
    protect(result, Functor(this, &TestCase::runTest));
  }
  protect(result, Functor(this, &TestCase::tearDown));
}
 \end{c++}
\end{leftbar}

\section{搬迁异常逻辑}

运用迪米特法则,将异常处理的逻辑从\ascii{TestCase::protect}搬迁至\ascii{TestResult::protect}。因为\ascii{TestResult::protect}依赖与\ascii{TestCaseFunctor}的抽象类型,而非具体的\ascii{TestCase},实现了\ascii{TestResult}对\ascii{TestCase}的解耦。在头文件中,公开\ascii{TestResult::protect},而私有\ascii{onFail, onError}。

\begin{leftbar}
 \begin{c++}[caption={\ttfamily{include/mars/core/TestResult.h}}]
struct TestCaseFunctor;

struct TestResult {
  TestResult();

  int failCount() const;
  int errorCount() const;

  bool protect(const TestCaseFunctor&);

private:
  void onFail();
  void onError();

private:
  int numOfFails;
  int numOfErrors;
};
 \end{c++}
\end{leftbar}

在实现文件中,搬迁异常处理逻辑至此处。

\begin{leftbar}
 \begin{c++}[caption={\ttfamily{src/mars/core/TestResult.cc}}]
#include "mars/core/TestResult.h"
#include "mars/core/internal/TestCaseFunctor.h"
#include "mars/except/AssertionError.h"

bool TestResult::protect(const TestCaseFunctor& f) {
  try {
    return f();
  } catch (const AssertionError&) {
    onFail();
  } catch (const std::exception&) {
    onError();
  } catch (...) {
    onError();
  }
  return false;
}
 \end{c++}
\end{leftbar}

\subsection{重构TestCase}

重构\ascii{TestCase},使其调用\ascii{TestResult::protect}完成异常处理的逻辑。测试通过后,删除既有的\ascii{TestCase::protect}。

\begin{leftbar}
 \begin{c++}[caption={\ttfamily{src/mars/core/TestCase.cc}}]
#include "mars/core/TestCase.h"
#include "mars/core/TestResult.h"
#include "mars/core/internal/TestCaseFunctor.h"

namespace {
  struct Functor : TestCaseFunctor {
    using Method = void(TestCase::*)();

    Functor(TestCase* self, Method method)
      : self(self), method(method) {
    }

  private:
    bool operator()() const override {
      (self->*method)();
      return true;
    }

  private:
    TestCase* self;
    Method method;
  };
}

void TestCase::run(TestResult& result) {
  if (result.protect(Functor(this, &TestCase::setUp))) {
    result.protect(Functor(this, &TestCase::runTest));
  }
  result.protect(Functor(this, &TestCase::tearDown));
}
 \end{c++}
\end{leftbar}

\subsection{改善表达力}

可以定义一个宏,改善\ascii{TestCase::run}对\ascii{TestResult::protect}调用的表达力。不忘初心,\ascii{TestCase::run}算法主干回归至最开始的简单逻辑。

\begin{leftbar}
 \begin{c++}[caption={\ttfamily{src/mars/core/TestCase.cc}}]
#define PROTECT(method) \
    result.protect(Functor(this, &TestCase::method))

void TestCase::run(TestResult& result) {
  if (PROTECT(setUp)) {
    PROTECT(runTest);
  }
  PROTECT(tearDown);
}
 \end{c++}
\end{leftbar}

\end{content}

