%%%%%%%%------------------------------------------------------------------------
%%%% xeCJK相关宏包

\usepackage{xltxtra,fontspec,xunicode}

%% \CJKsetecglue{\hskip 0.15em plus 0.05em minus 0.05em}
%% slanfont: 允许斜体
%% boldfont: 允许粗体
%% CJKnormalspaces: 仅忽略汉字之间的空白,但保留中英文之间的空白。 
%% CJKchecksingle: 避免单个汉字单独占一行。
\usepackage[slantfont, boldfont]{xeCJK} 
% \usepackage{ctex}

%% 针对中文进行断行
\XeTeXlinebreaklocale "zh"             

%% 给予TeX断行一定自由度
\XeTeXlinebreakskip = 0pt plus 1pt minus 0.1pt

%%%% xeCJK设置结束                                       
%%%%%%%%------------------------------------------------------------------------

%%%%%%%%------------------------------------------------------------------------
%%%% xeCJK字体设置

%% 设置中文标点样式,支持quanjiao、banjiao、kaiming等多种方式
\punctstyle{quanjiao}                                        
                                                     
%% 设置缺省中文字体
\setCJKmainfont[BoldFont={Adobe Heiti Std}, ItalicFont={Adobe Kaiti Std}]{Adobe Song Std}   %  FZBaoSongZ04
%% 设置中文无衬线字体
\setCJKsansfont[BoldFont={Adobe Heiti Std}, ItalicFont={Adobe Kaiti Std}]{Adobe Kaiti Std}  
%% 设置等宽字体
\setCJKmonofont{Adobe Heiti Std}                            
%\setCJKmonofont{Monaco}                            

%% 英文衬线字体
\setmainfont{Lucida Bright}                                  
%% 英文等宽字体
%\setmonofont{Courier}
\setmonofont{Monaco}                             
%\setmonofont{Consolas}                              
%% 英文无衬线字体
\setsansfont{Optima}                                   

%% 定义新字体
\setCJKfamilyfont{song}{Adobe Song Std}                     
\setCJKfamilyfont{kai}{Adobe Kaiti Std}
\setCJKfamilyfont{hei}{Adobe Heiti Std}
\setCJKfamilyfont{fangsong}{Adobe Song Std}
\setCJKfamilyfont{lisu}{LiShu}
\setCJKfamilyfont{youyuan}{Adobe Kaiti Std}

%%自定义英文字体
\newfontfamily\couriernew{Lucida Grande}
\newfontfamily\optima{Optima}
\newfontfamily\lucida{Lucida Bright}

\newcommand{\ascii}[1]{{\sffamily #1}}
\newcommand{\speak}[1]{{\itshape #1}}
\renewcommand{\emph}[1]{{\hei #1}}

%% 自定义宋体
\newcommand{\song}{\CJKfamily{song}}                       
%% 自定义楷体
\newcommand{\kai}{\CJKfamily{kai}}                         
%% 自定义黑体
\newcommand{\hei}{\CJKfamily{hei}}                         
%% 自定义仿宋体
\newcommand{\fangsong}{\CJKfamily{fangsong}}               
%% 自定义隶书
\newcommand{\lisu}{\CJKfamily{lisu}}                       
%% 自定义幼圆
\newcommand{\youyuan}{\CJKfamily{youyuan}}                 

%%%% xeCJK字体设置结束
%%%%%%%%------------------------------------------------------------------------

%%%%%%%%------------------------------------------------------------------------
%%%% 一些关于中文文档的重定义

%% 数学公式定理的重定义

\newtheorem{example}{例}[section]                                   
\newtheorem{algorithm}{算法}
%% 按section编号
\newtheorem{theorem}{定理}[section]                         
\newtheorem{definition}{定义}
\newtheorem{axiom}{公理}
\newtheorem{property}{性质}
\newtheorem{proposition}{命题}
\newtheorem{lemma}{引理}
\newtheorem{corollary}{推论}
\newtheorem{condition}{条件}
\newtheorem{conclusion}{结论}
\newtheorem{assumption}{假设}

\newtheorem{principle}{原则}[section]
\newtheorem{regulation}{规则}[section]
\newtheorem{advise}{建议}[section]
\newtheorem{concept}{概念}[section]

\usepackage{titlesec}

\renewcommand{\partname}{}
\renewcommand{\thepart}{第\Roman{part}部分}

%% 章节等名称重定义
\renewcommand{\contentsname}{目录}
%\renewcommand{\abstractname}{摘要}
\renewcommand{\indexname}{索引}
\renewcommand{\listfigurename}{插图目录}
\renewcommand{\listtablename}{表格目录}
\renewcommand{\figurename}{图}
\renewcommand{\tablename}{表}
\renewcommand{\appendixname}{附录}
\renewcommand{\appendixpagename}{附录}
\renewcommand{\appendixtocname}{附录}
%\renewcommand\refname{参考文献} 

%%设置内容环境
\newenvironment{content}{%
  \setlength{\parskip}{0.5\baselineskip}
  \begin{spacing}{1.5}
}{%
  \end{spacing}
  \setlength{\parskip}{-0.5\baselineskip}
  \vskip -0.5\baselineskip
}

% 插入小段故事的语法:
% \begin{story}
%   \begin{center}
%     \inlinetitle{分水岭}
%   \end{center}
% \end{story}
\newenvironment{story}
{
  \setlength{\parskip}{0.5\baselineskip}
  \hbox to \textwidth{\hfil\rule{\linewidth}{1.0mm}\hfil}
  \begin{spacing}{1.0}
}{%
  \end{spacing}
  \hbox to \textwidth{\hfil\rule{\linewidth}{1.0mm}\hfil}
  \setlength{\parskip}{-0.5\baselineskip}
  \vskip -0.5\baselineskip
}

%% 设置chapter、section与subsection的格式
%\titleformat{\chapter}[display]{\flushright\yihao}{\thechapter{}}{1em}{\textbf}
\titleformat{\section}[block]{\flushleft\sanhao}{\optima{\thesection}}{1em}{\textbf}
\titleformat{\subsection}{\sihao}{\optima{\thesubsection}}{0.5em}{\textbf}
\titleformat{\subsubsection}{\xiaosi}{\thesubsubsection}{0.5em}{\textbf}

%\titlespacing{\chapter}{0pt}{0pt}{-\baselineskip}
\titlespacing{\section}{0pt}{0pt}{0\baselineskip}
\titlespacing{\subsection}{0pt}{0.5\baselineskip}{0\baselineskip}

%% 设置章格式
\usepackage{quotchap}

\renewcommand\chapterheadstartvskip{
   \vspace*{-5\baselineskip}
}

\renewcommand\chapterheadendvskip{
   \vspace*{0.5\baselineskip}
}

\usepackage{helvet}
\renewcommand\sectfont{\rmfamily\bfseries}

\newcommand\refig[1]{{\itshape \figurename\ascii{\ref{fig:#1}(第\pageref{fig:#1}页)}}}
\newcommand\reftbl[1]{{\itshape \tablename\ascii{\ref{tbl:#1}(第\pageref{tbl:#1}页)}}}

\renewcommand{\footnoterule}{\vspace*{3pt}%
  \hrule width 0.382\textwidth height 0.4pt \vspace*{2.6pt}}

% Remark
\newenvironment{remark}{\par\vskip10pt\footnotesize\itshape % Vertical white space above the remark and smaller font size
\begin{list}{}{
\leftmargin=35pt % Indentation on the left
\rightmargin=25pt}\item\ignorespaces % Indentation on the right
\makebox[-2.5pt]{\begin{tikzpicture}[overlay]
\node[draw=red!60,line width=1pt,circle,fill=red!25,font=\sffamily\bfseries,inner sep=2pt,outer sep=0pt] at (-15pt,0pt){\textcolor{red}{R}};\end{tikzpicture}}
\advance\baselineskip -1pt}{\end{list}\vskip5pt}

%%%% 中文重定义结束
%%%%%%%%------------------------------------------------------------------------
