%%%% fancyhdr设置页眉页脚
%% 页眉页脚宏包
\usepackage{fancyhdr}

%% 页眉页脚风格
\pagestyle{fancy}

%%这两行代码是修改\leftmark和\rightmark的经典模式
\renewcommand{\chaptermark}[1]{\markboth{{\hei {第\thechapter{}章}}\hspace 1  #1}{}}
\renewcommand{\sectionmark}[1]{\markright{\thesection{} #1}}

%% 清空当前页眉页脚的默认设置
\fancyhf{}

%\fancyhead[L]{\scriptsize \fangsong \ascii{ZTE}中兴}
%\fancyhead[R]{\scriptsize \fangsong 内部公开}

%\fancyhead[CE]{\scriptsize \fangsong \leftmark}
%\fancyhead[CO]{\scriptsize \fangsong \rightmark}

%\fancyfoot[RO, LE]{\scriptsize \thepage}
%\fancyfoot[C]{\scriptsize \fangsong 本文中的所有信息均为中兴通讯股份有限公司内部信息,不得向外传播}

\renewcommand{\headrulewidth}{0.4pt}
\renewcommand{\footrulewidth}{0.4pt}

%第{\couriernew\thechapter{}}章
%%下面开始修改页眉和页脚
\fancyhead[RE]{\fangsong \leftmark}
\fancyhead[LO]{\fangsong \rightmark}
\fancyhead[RO, LE]{\small \thepage}
\fancypagestyle{plain}{%
  \fancyhead{} % get rid of headers
  \renewcommand{\headrulewidth}{0pt} % and the line.
}

%%定义空白页面
\makeatletter
\def\cleardoublepage{\clearpage\if@twoside \ifodd\c@page\else
  \hbox{}
  \vspace*{\fill}
  \begin{center}
   {\sffamily\large}
   \end{center}
   \vspace{\fill}
   \thispagestyle{empty}
   \newpage
   \if@twocolumn\hbox{}\newpage\fi\fi\fi}
\makeatother

\makeatletter
\def\cleardedicatepage{\clearpage
  \hbox{}
  \vspace*{\fill}
  \begin{center}
   {\sffamily\Large 献给我的女儿刘楚溪}
   \end{center}
   \vspace{\fill}
   \thispagestyle{empty}
   \newpage
   \if@twocolumn\hbox{}\newpage\fi}
\makeatother

%% 有时会出现\headheight too small的warning
\setlength{\headheight}{15pt}
